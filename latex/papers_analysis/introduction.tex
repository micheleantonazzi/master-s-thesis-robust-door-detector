\section{Introduction}

The phenotype is a term used in genetics to define the set of observable
traits in an organism. It can be seen as the result of three factors:
the expression of the genetic code of an organism, its interaction with
the environment, and some casual events which can occur during its life.
In particular, the main aspect is the gene expression, which is the
process by which the information from a gene is used to synthesize all
functional gene products, like RNA or proteins, essential for the life
of all organisms and many viruses. The first step to define the gene
expression is to study the DNA, which is the essential molecule that
characterizes the organisms in general. It contains all the genetic
instructions which regulate this process. Starting from its structure,
the DNA is composed of two chains of nucleotides, combined to compose a
double helix. This molecule can be divided into subsequences of
nucleotides and each of them has a specific task. These sequences can be
grouped into two sets according to their main purpose: the coding and
non-coding regions. The coding regions (the gene) contain the
information to synthesize some gene products and the non-coding regions
are responsible to help, regulate or modify the creation process of a
specific gene product. The gene expression involves both these regions
of the DNA and it is composed of two main processes: transcription and
translation. Through the transcription process, a particular DNA coding
region is copied inside the messenger RNA (mRNA) thanks to some
proteins, the RNA polymerase and other transcription factors, specific
for each gene product. In particular, RNA polymerase binds to a
particular region of DNA, called the promoter, and activates the
transcription process. The initial version of mRNA contains redundant
information and it consists of alternating segments called exons (coding
regions) and introns (redundant regions). The RNA splicing method is
applied to remove the introns and to produce the final mRNA sequence.
After that, the mRNA goes out from the cell nucleus, and then the
translation process starts. The mRNA sequence is divided into small
groups of 3 nucleotides. Each of these triplets, also known as codon,
corresponds to a specific amino acid. The result sequence of amino acids
forms the final gene product. It is important to specify that a single
coding region can generate many different products and the non-coding
region play a big role in gene expression. There are two main types of
non-coding regions: trans-regulatory elements (TREs) and cis-regulatory
elements (CREs). TREs are particular types of genes that may modify or
regulate the expression of other genes, often encoding transcription
factors. Other non-coding regions, the CREs, are close to the genes that
they regulate by bindings to the transcription factor. In particular,
enhancers and silencers interact with promoters, through RNA polymerase
and other transcription factors, to influence or repress, respectively,
the gene expression. There are hundreds of different cell types in an
organism despite they share the same DNA. This means that the gene
expression is different according to the cell line and the type of
product to be synthesized. In particular, the DNA may be used by the
cell in very different ways and its regions may be active or inactive,
producing a different level of expression for each gene. Determine the
activation of the regions is a very important task in biology, it can be
useful to determine the phenotype expressed by an organism or to control
the gene expression in a specific cell line or, again, to understand
better the complex interaction between the DNA and the transcription
factors. Besides, this technique can help during the diagnosis, to
determine the current or future pathology or to find the best therapy
according to the genetic characteristics of the patient. However,
determine if a DNA region is active or not is very hard and expensive:
the amount of data to collect is huge and its analysis is very complex.
The new information technologies can help to simplify this task,
supervised machine learning in particular. The idea is to train a
learning machine using some examples, DNA regions labeled as active and
inactive, so that it can predict the status of an unknown region
considering only its specific features. In the literature, a lot of
methods are proposed. In {[}1{]} it is used a deep feedforward neural
network to predict active enhancer, active promoter, active exon,
inactive enhancer, inactive promoter, inactive exon, and uncharacterized
regions. Another method, DeepEnhancer {[}2{]}, uses CNN to find
enhancers and specialize in the learning process on the cell line using
the transfer learning technique. Moreover, Basset {[}3{]} is an
open-source system that applies CNN to learn simultaneously the relevant
DNA sequence motifs and the regulatory logic to determine cell-specific
DNA accessibility. Following work, Basenji {[}4{]}, modify the previous
system to identify promoters and the relative regulatory elements and
synthesize their content to make more precise gene expression
predictions.