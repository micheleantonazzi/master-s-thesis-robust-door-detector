\newpage
\chapter{Abstract}

Mobile robots are active agents that operate interacting with the real world. To successfully execute the assigned task, a mobile robot has to build an abstract model of the environment in which it operates. Considering indoor scenes, doors are crucial features that a robot can acquire to make its environment's model more informative. The capabilities to detect doors, called \emph{doors detection}, can help robots to safely navigate in indoor environments, by improving their planning abilities and navigation strategies. The goal of this thesis is to propose a doors detector for autonomous mobile robots. We approach the problem of detecting doors as a visual object detection task translated in a robotic context. We develop the doors detector using DETR, a deep end-to-end model that performs object detection exploiting the power of a CNN backbone and a Transformer. Considering the typical deployment scenario of a mobile agent, in which a robot works in a single environment for a long time, we propose a technique to increase the performance of a general doors detector. By considering the \textit{wayfinding} principle, we argue that a single environment presents a coherent visual aspect. Following this intuition, also doors are similar in a single scene. The proposed approach, called \textit{one-shot incremental learning}, aims to specialize the module that finds doors with a few new examples to increase its performance in a precise environment.  We also investigate the necessary amount of new data (not included in the initial training phase) to obtain a significant performance improvement. Applying Deep Learning to Robotics generates a lot of challenges and open problems that are not completely addressed by the Computer Vision community. First of all, the well-known visual datasets (such as Microsoft COCO or Pascal VOC), are not acquired following an exploration strategy of a real autonomous agent and do not contain negative images (examples without objects of interest). This thesis offers a method to acquire a visual dataset in batch by emulating a real exploration strategy of a mobile robot. Furthermore, we propose an evaluation metric to measure the model's performance with negative images. Our results show that DETR can be used to detect doors even when trained with a smaller dataset than COCO. Furthermore, we demonstrate that the one-shot incremental learning paradigm increases the model's performance considering a single environment not considered during the initial training phase.

