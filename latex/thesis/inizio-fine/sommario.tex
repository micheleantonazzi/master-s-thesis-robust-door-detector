\newpage
\chapter{Abstract}

Mobile robots are agents that operate by physically interacting with the real world. To successfully execute the assigned task, a mobile robot has to build an abstract model of the environment in which it operates that represents all features of interest for its activities. Considering indoor scenes, the location of doors are crucial features that a robot should have in its environment's model. The capabilities to detect doors, called \emph{doors detection}, can help robots to safely navigate in indoor environments, by improving their planning abilities and navigation strategies. The goal of this thesis is to propose a doors detector for autonomous mobile robots. We approach the problem of detecting doors as a visual object detection task translated in a robotic context. We develop the doors detector using DETR, a deep end-to-end model that performs object detection exploiting the power of a CNN backbone and a Transformer. Considering the typical deployment scenario of a mobile agent, in which a robot works in a single environment for a long time, we propose a technique to increase the performance of a general doors detector. By considering the \textit{wayfinding} principle, we argue that a single environment presents a coherent visual aspect. Following this intuition, also doors are similar in a single scene. The proposed approach, called \textit{one-shot incremental learning}, aims to specialize the module that finds doors with a few new examples to increase its performance in a precise environment. Applying Deep Learning to Robotics generates a lot of challenges and open problems that are not completely addressed by the Computer Vision community. As an example, the well-known visual datasets (used for training neural networks) contain images that are acquired from a point of view very different from those of an autonomous robot. To solve such an issue, this thesis offers a method to acquire a visual dataset in batch by emulating a real exploration strategy of a mobile robot. Our results show that our framework can successfully detect doors even when trained with a small dataset. We also demonstrate that the one-shot incremental learning paradigm increases the model's performance on the robot's own operational environment.

