
% !TEX encoding = UTF-8
% !TEX TS-program = pdflatex
% !TEX root = tesi.tex
% !TEX spellcheck = it-IT

\documentclass[11pt,                    % corpo del font principale
               a4paper,                 % carta A4
               twoside,                 % impagina per fronte-retro
               openright,               % inizio capitoli a destra
               english]{book}    

%**************************************************************
% Importazione package
%************************************************************** 

%\usepackage{amsmath,amssymb,amsthm}    % matematica

\usepackage[T1]{fontenc}                % codifica dei font:
                                        % NOTA BENE! richiede una distribuzione *completa* di LaTeX

\usepackage[utf8]{inputenc}             % codifica di input; anche [latin1] va bene
                                        % NOTA BENE! va accordata con le preferenze dell'editor

\usepackage[english]{babel}    % per scrivere in italiano e in inglese;
                                        % l'ultima lingua (l'italiano) risulta predefinita

\usepackage{bookmark}                   % segnalibri

\usepackage{caption}                    % didascalie

\usepackage{chngpage,calc}              % centra il frontespizio

\usepackage{csquotes}                   % gestisce automaticamente i caratteri (")

\usepackage{emptypage}                  % pagine vuote senza testatina e piede di pagina

\usepackage{epigraph}			% per epigrafi

\usepackage{eurosym}                    % simbolo dell'euro

%\usepackage{indentfirst}               % rientra il primo paragrafo di ogni sezione

\usepackage{graphicx}                   % immagini

\usepackage{hyperref}                   % collegamenti ipertestuali

%\usepackage[binding=5mm]{layaureo}      % margini ottimizzati per l'A4; rilegatura di 5 mm

\usepackage{listings}                   % codici

\usepackage{microtype}                  % microtipografia

\usepackage{mparhack,fixltx2e,relsize}  % finezze tipografiche

\usepackage{nameref}                    % visualizza nome dei riferimenti                                      

\usepackage[font=small]{quoting}        % citazioni

\usepackage{subfig}                     % sottofigure, sottotabelle

\usepackage[italian]{varioref}          % riferimenti completi della pagina

\usepackage[dvipsnames]{xcolor}         % colori

\usepackage{booktabs}                   % tabelle                                       
\usepackage{tabularx}                   % tabelle di larghezza prefissata                                    
\usepackage{longtable}                  % tabelle su più pagine                                        
\usepackage{ltxtable}                   % tabelle su più pagine e adattabili in larghezza

\usepackage[toc, acronym]{glossaries}   % glossario
                                        % per includerlo nel documento bisogna:
                                        % 1. compilare una prima volta tesi.tex;
                                        % 2. eseguire: makeindex -s tesi.ist -t tesi.glg -o tesi.gls tesi.glo
                                        % 3. eseguire: makeindex -s tesi.ist -t tesi.alg -o tesi.acr tesi.acn
                                        % 4. compilare due volte tesi.tex.

\usepackage[style=numeric-comp,useprefix,hyperref,backend=bibtex]{biblatex}
                                        % eccellente pacchetto per la bibliografia; 
                                        % produce uno stile di citazione autore-anno; 
                                        % lo stile "numeric-comp" produce riferimenti numerici
                                        % per includerlo nel documento bisogna:
                                        % 1. compilare una prima volta tesi.tex;
                                        % 2. eseguire: biber tesi
                                        % 3. compilare ancora tesi.tex.
\addbibresource{bibliografia.bib}
\input{tesi-config}                     % file con le impostazioni personali
\usepackage{fancyhdr}


\usepackage[top=3.6cm,bottom=3.5cm,outer=3.1cm, inner=3.7cm, twoside, a4paper]{geometry}
%-------------- intestazione e piè di pagina
\pagestyle{fancy}
\fancyhf{}
\fancyhead[LE,RO]{\nouppercase\leftmark}
\fancyfoot[CE,CO]{\thepage}


%---------------- elenchi puntati
\renewcommand{\labelitemi}{\textbullet}
\renewcommand{\labelitemii}{\textendash}
\renewcommand{\labelitemiii}{\textasteriskcentered}



\begin{document}
%**************************************************************
% Materiale iniziale
%**************************************************************
\frontmatter
\pagenumbering{gobble} 
% !TEX encoding = UTF-8
% !TEX TS-program = pdflatex
% !TEX root = ../tesi.tex

%**************************************************************
% Frontespizio 
%**************************************************************
\begin{titlepage}

\begin{center}

\begin{LARGE}
\textbf{\myUni}\\
\end{LARGE}

\vspace{10pt}

\begin{Large}
\textsc{\myDepartment}\\
\end{Large}

\vspace{10pt}

\begin{Large}
\textsc{\myFaculty}\\
\end{Large}

\vspace{30pt}
\begin{figure}[htbp]
\begin{center}
\includegraphics[height=6cm]{images/unimilogo.png}
\end{center}
\end{figure}
\vspace{10pt} 

\begin{LARGE}
\begin{center}
\textbf{\myTitle}\\
\end{center}
\end{LARGE}

\vspace{8pt} 

\begin{large}
\textsl{\myDegree}\\
\end{large}

\vspace{30pt} 

\begin{large}
\begin{flushleft}
\textit{Relatore}\\ 
\vspace{3pt} 
\profTitle\:\myProf
\end{flushleft}

\begin{flushleft}
	\textit{Correlatore}\\ 
	\vspace{3pt} 
	\correlatoreTitle\:\myCorrelatore
\end{flushleft}

\begin{flushright}
\textit{Laureando}\\ 
\vspace{3pt}  
\myName\\
Mat. 936431
\end{flushright}
\end{large}

\vspace{40pt}

\line(1, 0){338} \\
\begin{normalsize}
\textsc{Anno Accademico \myAA}
\end{normalsize}

\end{center}
\end{titlepage} 
\newpage
\input{inizio-fine/colophon}
% !TEX encoding = UTF-8
% !TEX TS-program = pdflatex
% !TEX root = ../tesi.tex

%**************************************************************
% Dedica
%**************************************************************
\cleardoublepage
\phantomsection
\thispagestyle{empty}
\pdfbookmark{Dedica}{Dedica}

\vspace*{3cm}

\begin{center}
``I traguardi della vita perdono di significato se non si hanno persone con cui condividerli.'' \\ \medskip
--- Un vecchio saggio   
\end{center}

\medskip

\begin{center}

\end{center}

\newpage
\chapter{Abstract}

Mobile robots are active agents that operate interacting with the real world. To successfully execute the assigned task, a mobile robot has to build an abstract model of the environment in which it operates. Considering indoor scenes, doors are crucial features that a robot can acquire to make its environment's model more informative. The capabilities to detect doors, called \emph{doors detection}, can help robots to safely navigate in indoor environments, by improving their planning abilities and navigation strategies. The goal of this thesis is to propose a doors detector for autonomous mobile robots. We approach the problem of detecting doors as a visual object detection task translated in a robotic context. We develop the doors detector using DETR, a deep end-to-end model that performs object detection exploiting the power of a CNN backbone and a Transformer. Considering the typical deployment scenario of a mobile agent, in which a robot works in a single environment for a long time, we propose a technique to increase the performance of a general doors detector. By considering the \textit{wayfinding} principle, we argue that a single environment presents a coherent visual aspect. Following this intuition, also doors are similar in a single scene. The proposed approach, called \textit{one-shot incremental learning}, aims to specialize the module that finds doors with a few new examples to increase its performance in a precise environment.  We also investigate the necessary amount of new data (not included in the initial training phase) to obtain a significant performance improvement. Applying Deep Learning to Robotics generates a lot of challenges and open problems that are not completely addressed by the Computer Vision community. First of all, the well-known visual datasets (such as Microsoft COCO or Pascal VOC), are not acquired following an exploration strategy of a real autonomous agent and do not contain negative images (examples without objects of interest). This thesis offers a method to acquire a visual dataset in batch by emulating a real exploration strategy of a mobile robot. Furthermore, we propose an evaluation metric to measure the model's performance with negative images. Our results show that DETR can be used to detect doors even when trained with a smaller dataset than COCO. Furthermore, we demonstrate that the one-shot incremental learning paradigm increases the model's performance considering a single environment not considered during the initial training phase.


% !TEX encoding = UTF-8
% !TEX TS-program = pdflatex
% !TEX root = ../tesi.tex

%**************************************************************
% Ringraziamenti
%**************************************************************
\cleardoublepage
\phantomsection
\pdfbookmark{Ringraziamenti}{ringraziamenti}

\bigskip

\begingroup
\let\clearpage\relax
\let\cleardoublepage\relax
\let\cleardoublepage\relax

\chapter*{Ringraziamenti}

\noindent \textit{Innanzitutto, vorrei esprimere la mia gratitudine al \profTitle  \myProf, relatore della mia tesi, per avermi seguito durante la stesura del documento.}\\

\noindent \textit{Un ringraziamento particolare va ai miei genitori che non hanno mai smesso di credere in me, mettendo sempre in primo piano la mia istruzione piuttosto che gli aspetti economici e logistici che essa comporta.}\\

\noindent \textit{Ho desiderio di ringraziare la mia ragazza che mi ha supportato durante questi anni di studio, facendomi sempre spuntare un sorriso nei momenti di difficoltà.}\\

\noindent \textit{Un ringraziamento va a tutti i miei amici che durante gli studi non mi hanno mai fatto mancare momenti di svago e divertimento.}\\

\noindent \textit{Un ultimo pensiero va ai colleghi di SICS S.r.l. per aver contribuito a rendere significativa e piacevole l'esperienza di stage. In particolare ringrazio Francesco Bazzerla per tutto l'aiuto fornitomi durante lo svolgimento del progetto.}\\
\bigskip

\noindent\textit{\myLocation, \myTime}
\hfill \myName

\endgroup


% !TEX encoding = UTF-8
% !TEX TS-program = pdflatex
% !TEX root = ../tesi.tex

%**************************************************************
% Indici
%**************************************************************
\cleardoublepage
\pdfbookmark{\contentsname}{tableofcontents}
\setcounter{tocdepth}{6}
\setcounter{secnumdepth}{15}
\tableofcontents
%\markboth{\contentsname}{\contentsname} 
\clearpage

\begingroup 

    %*******************************************************
    % Elenco delle figure
    %*******************************************************    
    \phantomsection
    \pdfbookmark{\listfigurename}{lof}
    \listoffigures

    \vspace*{8ex}

    %*******************************************************
    % Elenco delle tabelle
    %*******************************************************
    \phantomsection
    \pdfbookmark{\listtablename}{lot}
    \listoftables
        
    \vspace*{8ex}
\endgroup

\cleardoublepage

\cleardoublepage

%**************************************************************
% Materiale principale
%**************************************************************
\mainmatter
% !TEX encoding = UTF-8
% !TEX TS-program = pdflatex
% !TEX root = ../tesi.tex

%**************************************************************
\hypertarget{Introduzione}{%
	\chapter{Introduzione}\label{header-n3}}             % Introduzione
\chapter{State of the art}
\label{capitolo2}
\thispagestyle{empty}

 - Introduco il tema del riconoscimento delle porte, citando qualche articolo e delineando una storia 
 - quando arrivo ai classificatori deep, inizio a parlare di deep learning e del forte impatto che sta avendo sulla robotica
 - detto questo mi collego alla object detection e cito le milestone principali 
 - qua inizio a parlare dei problemi che deep learning e robotica implicano (survey)
 - 
 \newline
 In this chapter, we present  
 
 
 \section{Doors Detection}
 Mobile robots are active agents that operate interacting with real world. To successfully execute the assigned task, a mobile robot has to build an abstract model of the environment in which it operates. Considering indoor scenes, doors are crucial features that a robot can acquire to make its environment's model more informative. 
 Smart vacuum cleaners, healthcare robots or intelligent housekeepers helps people in their daily task. Usually, the tasks assigned these autonomous agents imply moving between rooms and dealing with doors.
 Doors detection can help these types of agents to safely navigate in indoor environments, by improving their reasoning abilities and navigation strategies. 
 
 
 \section{title}
    They have to perceive environmental features, build an abstract model of the real world, decide and plan future actions, and finally act to fulfill the assigned task. Operating in real world, autonomous agents operate in uncontrolled condition i Using sensors, an autonomous agent acquire data directly to the environment in order to build an abstract model of the real world in which it operates. (qui si potrebbe riportare qualcosa su laser, lidar ecc ma non centra molto con il nostro lavoro). Traditional sensors used in robotics, like ultrasonic sensor or LiDAR, converts physical signals (sound and light respectively) into distance measurements. An autonomous agent can aggregate these data using    information gets environmental characteristics using physical signals (sound or light respectively). Using this data, a mobile robot can and  To get an extensive representation of the environment, a mobile robot can acquire and analyze RGB frames from a camera.  in which  the robot operates 
 
 Mobile Robotics and Computer Vision are two disciplines that are growing rapidly in the recent years. Modern technologies and an intensive research activity have brought promising results in these fields.
 Nevertheless, Mobile Robotics and Computer Vision   
             % Processi
\chapter{Impostazione del problema di ricerca}
\label{capitolo3}
\thispagestyle{empty}

\begin{quotation}
{\footnotesize
\noindent{\emph{``''}}
\begin{flushright}
Altrimenti ci arrabbiamo
\end{flushright}
}
\end{quotation}
\vspace{0.5cm}

\noindent In questa sezione si deve descrivere l'obiettivo della ricerca, le problematiche affrontate ed eventuali definizioni preliminari nel caso la tesi sia di carattere teorico.             % Kick-Off
\chapter{System Detail}
\label{sec:chapter4}
In this chapter, we define in detail the system we develop to solve to implement the method described Sec. \ref{sec:solution}. The system is composed of five main modules and each of them addresses a specific requirement imposed by our solution. We report the functionalities of each module, describing also the algorithms we implement and the technologies used for their construction. These modules are the following:

\begin{enumerate}
	\item \textbf{Doors Detector:} this module implements the main requirement of this thesis: the development of a doors detector for indoor autonomous robots. We approach the problem of finding doors as an object detection task, that we address using a deep learning technique. Doors detector is implemented through DETR \cite{detr}, a deep end-to-end architecture to perform object detection. To better understand how DETR works, we provide an overview of the principal machine learning and deep learning principles. Finally, we described in detail the DETR's architecture. 
	\item \textbf{Simulation Environment:} the second part of our system is the simulation environment used to collect the dataset. As described in Sec. \ref{sec:importanceofsimulation}, simulation is widely used in robotic applications, so we employ this technique to collect the dataset for training the doors detector. The acquisition of the doors dataset solves the lack of suitable datasets for robotic applications, as explained in Sec. \ref{sec:goals}. We choose Gibson \cite{gibson} as the virtualization framework and Matterport3D \cite{matterport} as the worlds' dataset. In the dedicated section, we describe the main functionalities of Gibson and Matterport3D. Then, we report the issues of these technologies that prevent correct and fast data collection. We finally explain the adopted solution to overcome these limitations: we developed a modified version of Gibson that implements a new simulation stage in which the robot can be placed freely in any location.
	\item \textbf{Pose Estimator:} this module emulates a possible navigation strategy used by a real autonomous agent and extracts  the locations from which the data points are collected. Thanks to this module unified with the new version of Gibson, we offer a method to easily acquire a visual dataset which models the \textit{open-set} conditions in which an autonomous agent operates. In this section, we report the underlying algorithms implemented in this component.
	\item \textbf{Dataset Manager:} another important aspect of the system provided by this thesis is the dataset. We report in detail by which data it is composed, describing also the labeling procedure. In addition, we mention the framework we developed for managing the dataset.
	\item  \textbf{Model Evaluator:} the last component of our system aims to evaluate the model's performance. As reported in Sec. \ref{sec:solution}, we propose a new metric that considers also the negative images (examples with no objects of interest). For this module, we describe the metric we propose and the procedure to obtain its results.
\end{enumerate}


\section{Doors Detector}
\label{sec:doors_detector}

The doors detector we propose is built with DETR \cite{detr}. The module we develop is public available on Github, in the main repository\footnote{The main thesis's repository: \url{https://github.com/micheleantonazzi/master-thesis-robust-door-detector}.} of this thesis inside the \textsf{doors-detector} package. In this section, we describe in detail the architecture and the loss functions of DETR.

\subsection{DETR}
\label{sec:detr}
The doors detector proposed by this thesis is based on DETR \cite{detr} (DEtection TRansformer), a novel deep approach to perform object detection.  In DETR, the object detection problem is modeled as a direct \textit{set prediction}, making the detection pipeline a simple end-to-end unified architecture. Modern detectors address this set prediction in an indirect way using hand-crafted algorithms based on a large set of proposals \cite{yolo} or anchors \cite{focalloss}. Their performances are significantly influenced by these post-processing steps to collapse near-duplicate predictions, like non-maximum suppression or anchor generation. The first important aspect of DETR  is its architecture (explained in Sec. \ref{sec:detrarchitecture}). DETR uses a Transformer to find complex relationships between features extracted in the same image. In this way, the model reasons about the whole image context without considering any form of prior knowledge about the task. The second important aspect to address the set prediction problem is the loss functions (described in \ref{sec:detrlosses}). DETR uses a set loss function that performs bipartite matching between predicted and ground-truth objects.

\subsubsection{DETR Architecture}
\label{sec:detrarchitecture}
The architecture of DETR (reported in Fig. \ref{fig:detrarchiecture}) consists of three main components: a backbone convolutional neural network (CNN) for features extraction, an encoder-decoder transformer to capture the relationships between the extracted features, and a simple feed-forward network (FFN) that makes the final detection prediction (the coordinates of the bounding boxes and their relative labels).

\begin{figure}[h!]
	\centering
	\includegraphics[width=\linewidth]{images/detrarchitecture.pdf}
	\caption{The architecture of DETR. It uses a conventional CNN backbone to learn a 2D representation of an input image. Then, this representation is flattened and encoded before being passed into the transformer encoder. The transformer decoder then takes as input a small fixed number of learned positional embeddings, called \textit{object queries}, and additionally attends to the encoder output. Finally, each output embedding of the decoder is processed by a  simple feed-forward network (FFN) that predicts either a detection (class and bounding box) or a ``no object'' class. Image from \cite{detr}.}
	\label{fig:detrarchiecture}
\end{figure}

\paragraph{CNN Backbone} The first element in DETR's architecture is a CNN backbone.  Ideally, any backbone can be used depending upon the complexity of the task to provide a low dimensional representation of the image extracting features from it. Starting from the initial image $x_{img} = \R^{3\times H_0\times W_0}$ (with 3 color channel and dimension $H_0$, $W_0$), a conventional CNN backbone generates a lower-resolution activation map $f \in \R^{C \times H \times W}$. Typical values used in this paper are $C = 2048$ and $H, W = \frac{H_0}{32}, \frac{W_0}{32}$. 

The authors use a ResNet (residual network) \cite{resnet} as CNN backbone, a neural network that solves the issue of vanishing gradient. This problem makes a deeper neural network more difficult to train and optimize: when a network's depth increases, accuracy gets saturated and then degrades rapidly. This unwanted situation is called \textit{degradation} (of training accuracy). Unexpectedly, such degradation is not caused by overfitting, and adding
more layers to a deep model leads to higher training error. The authors address the degradation problem by introducing a \textit{deep residual learning} framework, commonly called ResNet. The proposed method is based on the insight that a few overlapping layers can fit a residual mapping instead of directly fitting a desired underlying mapping. This principle is formalized as follows. Let $\mathcal{H}(x)$ be the desired underlying mapping fit by a few stacked layers (not necessarily the entire net), where $x$ denotes the inputs of the first layer. Since multiple non-linear layers can approximate non-linear functions, then  the same layers can approximate the residual functions $\mathcal{H}(x) - x$. The authors of ResNet \cite{resnet} let these layers approximate a residual function
$\mathcal{F}(x) := \mathcal{H}(x) - x$, so the original function becomes
$\mathcal{F}(x)+x$. The authors hypothesize that it is easier to optimize the residual mapping than to optimize the original unreferenced mapping.  If the optimal function is closer to an identity
mapping than to a zero mapping, it should be easier for the
solver to find the perturbations with reference to an identity
mapping, than to learn the function as a new one. This means that subsequent blocks in our network are thus responsible for fine-tuning the output of a previous block, instead of having to generate the desired output from scratch.

\begin{figure}[h!]
	\centering
	\includegraphics[width=\linewidth]{images/residualblock.pdf}
	\caption{Residual learning: a building block. The formulation of $\mathcal{F}(x)+x$ can be realized with a ``shortcut connection'', that skips one or more layers. Image from \cite{resnet}.}
	\label{fig:resblock}
\end{figure}

\paragraph{Transformer Encoder-Decoder} The second part of DETR is a Transformer \cite{transformer}, a sequence-to-sequence (Seq2Seq) architecture that transforms a given sequence of elements, such as the sequence of words in a sentence, into another sequence. Seq2Seq models are particularly useful for natural language process tasks (NLP), text classification, machine translation, and
question answering. Among their salient benefits, Transformers enable modeling long dependencies between input sequence elements, support parallel processing, and require minimal inductive biases for their design. Recent studies \cite{surveytransformer} demonstrate the application of Transformer in Computer Vision. By using a Transformer model, DETR globally reasons about all objects together using pair-wise relations between them and being able to use the whole image as context. In this way, DETR predicts (in a single pass) a set of objects and
models their relations. Since DETR performs object detection, the Transformer's input is a sequence of features extracted from an image.

The Transformer model, proposed by \citeauthor{transformer} \cite{transformer}, is the first transduction model relying entirely on self-attention mechanism to compute representations of its input and output without using RNNs or convolutions. Self-attention, sometimes called intra-attention, is an attention mechanism that relates every single element in a sequence with the other elements and finally computes a representation of the entire sequence. In other words, the attention mechanism decides at each step which parts of the sequence are important by assigning a weight to each element. The Transformer architecture (Fig. \ref{fig:transformerarc}) is based on an encoder-decoder structure. In a general way, the encoder maps an input sequence of symbol representations $(x_1, ..., x_n)$ into a sequence of continuous representations $z = (z_1, ..., z_n)$. Given $z$, the decoder then generates an output sequence $(y_1, ..., y_m)$ of symbols one element at a time.

More specifically, the Transformer's encoder  is composed of a stack of $N = 6$ identical layers. Each layer has two sub-layers: the first is a multi-head self-attention mechanism and the second is a simple, position-wise fully connected feed-forward network. The authors add a residual connection (like ResNet \cite{resnet}) around each sub-layer, followed by layer normalization. The output of each sub-layer is $LayerNorm\big(x + Sublayer(x)\big)$. The decoder is also composed of a stack of $N = 6$ identical layers. Each of them has the same sub-layers of the encoder with a third one that performs multi-head attention over the output of the encoder stack. The authors modify the self-attention
sub-layer in the decoder to prevent positions from attending to subsequent positions. This masking, combined with fact that the output embeddings are offset by one position, ensures that the predictions for the position $i$ can depend only on the known outputs at positions less than $i$.

\begin{figure}[h!]
	\centering
	\includegraphics[width=0.8\linewidth]{images/transformerarchitecture.png}
	\caption{The Transformer model architecture. Image from \cite{transformer}.}
	\label{fig:transformerarc}
\end{figure}

DETR \cite{detr} uses a standard Transformer developed for natural language process as proposed in \cite{transformer}. The high-level activation map $f$ extracted from the CNN backbone is rescaled from $C$ to a smaller dimension $d$ by a $1\times1$ convolution. The new feature map fed into the Transformer's encoder is $z_0 \in \R^{d, H, W}$. Since the encoder requires a sequence as input, the feature map $z_0$ is collapsed the spatial into one dimension, resulting in a $d\times HW$ feature map. The decoder transforms $N$ embeddings of size $d$.  The difference with the original Transformer \cite{transformer} is that DETR decodes the $N$ objects in parallel at each decoder layer. These input embeddings (called \textit{object queries}) are positional encoding vectors learned by the model during the training phase. They are passed to the input of each attention layer. Since the decoder is permutation-invariant, the $N$ input embeddings must be different to produce different results. Each object query is transformed into an output embedding by the decoder. The number of object queries is a hyper-parameter and it must be greater than the quantity of different objects in an image.

\paragraph{Feed-Forward Networks} The $N$ object queries produced by the decoder are  independently classified into box coordinates and class labels by
a feed-forward network, resulting in $N$ final object predictions. The final predictor is composed of a 3-layer perceptron and a linear projection layer. The perceptron predicts the normalized center coordinates, height, and width of the bounding boxes while the linear layer predicts the class labels. Since DETR predicts a
fixed-size set of $N$ bounding boxes, where $N$ is much larger than the
actual number of objects of interest in an image, an additional special class label $\varnothing$ is used to represent that no object is detected within a slot (it indicates the ``background'' class).

\subsubsection{DETR Loss Functions}
\label{sec:detrlosses}

DETR \cite{detr} simplifies the detection pipeline by dropping multiple hand-designed components that encode prior knowledge, like spatial anchors or non-maximal suppression. To address set prediction in a fully end-to-end way, DETR uses a novel loss function, called \textit{object detection set prediction loss}, that produces
optimal bipartite matching between predicted and ground-truth objects, considering both the class labels and the bounding boxes.

\paragraph{Object Detection Set Prediction Loss} DETR \cite{detr} infers a fixed-size set of $N$ predictions through the $N$ object queries produced by the Transformer's decoder. It is important that $N$ is significantly larger than the maximum number of objects in every image. One of the main difficulties of training is to score predicted objects, by considering the class, position, and size, with respect to the ground-truth. The authors propose a loss function that produces
an optimal bipartite matching between predicted and ground-truth objects considering object-specific (bounding box) losses. 

\begin{figure}[h!]
	\centering
	\includegraphics[width=0.9\linewidth]{images/bipartitematching.png}
	\caption{A matching in a bipartite graph. It consists of a set of edges chosen in such a way that no two edges share an endpoint. The loss function $\hat \sigma$ (Eq. \ref{formula:bipartitematching}) finds the best match with the lowest cost between the predicted bounding boxes and the ground-truth objects.}
	\label{}
\end{figure}

Let $y$ be the ground-truth set of objects, and $\hat y = \{\hat y_i\}^{N}_{i = 1}$ the set of $N$ predictions. The bipartite matching between these two sets is the best permutation $\hat \sigma \in \mathcal{G}_N$ of $N$ elements with the lowest cost:

\begin{equation}
\label{formula:bipartitematching}
\hat \sigma = \argmin_{\sigma \in \mathcal{G}_N} \sum_{i}^{N} \mathcal{L}_{match}(y_i, \hat y_{\sigma(i)}),
\end{equation}
where $\mathcal{L}_{match}(y_i, \hat y_{\sigma(i)})$ is a pair-wise \textit{matching cost} between  ground-truth $y_i$ and a prediction with index $\sigma(i)$. This optimal assignment is computed efficiently
with the Hungarian algorithm \cite{hungarian}. The matching cost takes into account both the class prediction and the similarity between predicted and the ground-truth bounding boxes. Each element $i$ can be seen as $y_i = (c_i
, b_i)$, where $c_i$
is the target class label (which may be $\varnothing$) and $b_i \in [0, 1]^{4}$
is a vector that defines ground-truth box coordinates and dimensions. For the
prediction with index $\sigma(i)$, the probability of class $c_i$ is defined as $ \hat p_{\sigma(i)}(c_i)$ and
the predicted box as $\hat b_\sigma(i)$. With this notation, the authors define the matching cost as

\begin{equation}
	\mathcal{L}_{match}(y_i, \hat y_{\sigma(i)}) = -\mathds{1}_{\{c \neq \varnothing\}}\hat p_{\sigma(i)}(c_i) + \mathds{1}_{\{c \neq \varnothing\}}\mathcal{L}_{box}(b_i, \hat b_{\sigma(i)}),
\end{equation}
which performs one-to-one matching for direct set prediction without duplicates. 

The next step is to compute the loss function, called \textit{Hungarian loss}, for all pairs matched in the previous step. It is a linear combination of a negative log-likelihood for class prediction and a box loss defined later:

\begin{equation}
\label{eq:detr_loss}
\mathcal{L}_{\text{Hungarian}}(y, \hat y) = \sum_{i = 1}^{N} \bigg [-\log \hat p_{\hat\sigma(i)}(c_i) + \mathds{1}_{\{c \neq \varnothing\}}\mathcal{L}_{box}(b_i, \hat b_{\hat\sigma(i)}) \bigg ]
\end{equation}
where $\hat \sigma$ is the optimal assignment computed in Eq. \ref{formula:bipartitematching}. In practice, the authors
down-weight the log-probability term when $c_i = \varnothing$ by a factor 10 to account for class imbalance. 

\paragraph{Bounding Box Loss} The second part of the matching cost in the Hungarian loss is $\mathcal{L}_{box}(\cdot, \cdot)$ that scores the bounding boxes. The authors of DETR \cite{detr} perform box predictions directly without any initial guesses using the $\ell_1$ loss. In our context, this loss measures the distance between the  ground-truth bounding box ($b_i$) and the best predicted box ($\hat b_{\sigma(i)}$) through the L1 norm:

\begin{equation}
\ell_1(b_i - \hat b_{\sigma(i)}) = ||b_i - \hat b_{\sigma(i)}||_1.
\end{equation}

While such an approach simplifies the implementation, the $\ell_1$ loss will have different scales for small and large boxes even if their relative error is similar. To mitigate this issue, the final $\mathcal{L}_{box}(\cdot, \cdot)$ is a linear combination of the $\ell_1$ loss and the generalized IoU loss \cite{generalizediou} $\mathcal{L}_{iou}(\cdot, \cdot)$, that is scale-invariant because it is the ration between the intersection and the union as between a predicted and a ground-truth bounding boxes. The final bounding box loss is:

\begin{equation}
\label{eq:bounding_box_loss}
\mathcal{L}_{box}(b_i, \hat b_{\sigma(i)}) = \lambda_{iou}\mathcal{L}_{iou}(b_i, \hat b_{\sigma(i)}) + \lambda_{L1}||b_i - \hat b_{\sigma(i)}||_1,
\end{equation}
where $\lambda_{iou}$ and $\lambda_{L1}$ are hyper-parameters.

\section{Simulation Environment}

In this section, it is described the method we propose to build the doors dataset. As mentioned in Sec. \ref{sec:importanceofsimulation}, efficiently collecting a large and heterogeneous visual dataset in the real world from a robot point of view is extremely expensive and time-consuming. The images should be captured from different points of view and illumination conditions to emulate the freedom of movement that characterizes an autonomous agent. Furthermore, the collection procedure must be performed in a large number of different scenes and building types, to well generalize the problem. The visual aspect of indoor environments changes a lot according to the building type, the internal design, and the furniture's position.

Following these requirements, we use simulation to compose the visual dataset used in this thesis as simulation allows to collect in an automated way images from different points of view in thousand of environment with a limited overhead. As anticipated in Sec. \ref{sec:solution}, we use Gibson \cite{gibson} as simulation environment and Matterport3D \cite{matterport} as worlds dataset. We describe these packages in the following sub-sections, reporting also the issue encountered with these technologies and our proposed solution to mitigate them.

\subsection{Gibson Environment}

In \citeyear{gibson}, \citeauthor{gibson} published a robotic simulation environment called Gibson \cite{gibson}. Gibson offers a real-world perception mechanism for active agents. By perceptual active agent, the authors refer to
an agent that receives a visual observation from the environment and accordingly effectuates a set of actions, such as locomotion or manipulation. A key question is \textit{where} this sensory observation should come from. Conventional Computer Vision datasets \cite{coco, imagenet} are static and passive, so not suitable for this purpose.  Similarly, learning in the real world is not the ideal scenario: the learning speed is extremely slow and the robots are often costly and fragile. Simulation can be a solution to mitigate these issues. 
The primary problems around this option are naturally around \textit{generalization
from simulation to real world}, so how to ensure that:
\begin{enumerate}
	\item \label{enum:generalizationtorealworld1} the semantic
	complexity of the simulated environment is comparable with the real-word, and
	\item \label{enum:generalizationtorealworld2} the rendered frames in simulation are closed enough to the images captured by a camera in the real world (photorealism).
\end{enumerate} 
The main goal of Gibson is to facilitate transferring the
models trained therein to real world. To overcame the first issue, Gibson offers a framework to virtualize environments scanned from the real world. Furthermore, Gibson simulates real agents that can interact with the virtual scene by respecting some physical constraints (e.g. collision and gravity). In addition, Gibson implements a mechanism to dissolve the differences between virtual renderings and what a real camera would produce (this mitigates the second problem). A neural network trained to fill the perceptual gap between rendered and real frames.

Gibson’s underlying database of spaces includes 572 full buildings composed of 1447 floors covering a total area of 211k $m^{2}$. Each space has a set of RGB panoramas with global camera poses and reconstructed 3D meshes. To include semantically annotated worlds, the authors also integrated 2D-3D-Semantic dataset \cite{stanford2d3d} and Matterport3D \cite{matterport} (used in this thesis) for easy use.

The Gibson's rendering engine takes a sparse set of RGB-D panoramas in the input and renders a new panorama from an arbitrary novel viewpoint. A ``view'' is a 6D camera pose of $x, y, z$ Cartesian coordinates and roll, pitch, yaw angles, denoted as $\theta, \phi, \gamma$. 

\begin{figure}[h!]
	\centering
	\includegraphics[width=\linewidth]{images/gibson_rendering_pipeline.pdf}
	\caption{The Gibson's rendering pipeline. Image from \cite{gibson}.}
	\label{fig:gibsonrenderingpipeline}
\end{figure}
At first, the given RGB-D panoramas are transformed into point clouds and each pixel is projected from equirectangular coordinates to Cartesian coordinates. For the desired target view $v_j =
(x_j , y_j , z_j , \theta_j, \phi_j, \gamma_j )$, there are chosen the nearest $k$ views in the
scene database, denoted as $v_{j,1}, v_{j,2}, ..., v_{j,k}$. The point cloud of each $v_{j,i}$ coordinate is transformed to $v_j$ coordinate with a rigid body transformation, and the final point cloud is projected onto an equirectangular image (Fig. \ref{fig:gibsonrenderingpipeline}a). Then, the points from all reference panoramas are aggregated to make a single panorama using a locally weighted mixture (Fig. \ref{fig:gibsonrenderingpipeline}b). To do this, the proposed approach calculates the point density for each spatial position (average number of points per pixel) of each panorama. Hence, the points in the aggregated panorama are adaptively selected from all views. Next, the authors perform a bilinear interpolation on the aggregated points of the final panorama
points in one equirectangular image to reduce the empty space between rendered pixels (Fig. \ref{fig:gibsonrenderingpipeline}c). Finally, the method uses a neural network, called $f$ or ``filler'', to fix artifacts and generate a more real-looking image given the output of geometric point cloud rendering (Fig. \ref{fig:gibsonrenderingpipeline}d). This deep model trains a network for making rendered frames look more like real images (forward function) as well as another network that makes real images look like renderings (backward function). The two functions are trained to produce equal outputs. The backward function resembles deployment-time corrective
glasses for the agent, so the authors of \cite{detr} call it \textit{Goggles}.

\subsection{Matterport3D}

In \citeyear{matterport}, \citeauthor{matterport} introduced Matterport3D, a large-scale RGB-D dataset of 90 digitizes real environments. The dataset comprises a set of 194,400 RGB-D images captured in 10,800 panoramas from 90 real-world scenes. It includes both depth and color $360^{\circ} $ panoramas for each viewpoint and provides camera poses that are globally consistent and aligned with a textured surface reconstruction. Furthermore, Matterport3D includes instance-level semantic segmentation into region and object categories.

In the Matterport data acquisition process, an operator captures
a set of panoramas uniformly spaced at approximately 2.5m
throughout the entire walkable floor plan of the environment (Fig. \ref{fig:matterport-panoramas}). It uses a camera system rig with three color and three depth cameras pointing slightly up, horizontal, and slightly down. At each panorama (acquisition point), the rig rotates around the direction of gravity to
6 distinct orientations, acquiring an HDR photo from each of the 3 RGB cameras for each orientation. The 3 depth cameras acquire data continuously as the rig rotates, which is integrated to synthesize a 1280x1024 depth image aligned
with each color image. The result for each panorama is 18 RGB-D images. 

\begin{figure}[h!]
	\centering
	\includegraphics[width=\textwidth]{images/panoramas.pdf}
	\caption{The viewpoints from which panoramas are captured. Image from \cite{matterport}.}
	\label{fig:matterport-panoramas}
\end{figure}

The authors also provide instance-level semantic annotations in 3D. The first step of the annotation process is to break down each building into region components by specifying the 3D spatial extent and semantic category label for
each room-like region. This is done using a simple interactive tool in which the annotator draws a 2D polygon on the floor plans for each region (Fig. \ref{fig:matterport-floor-annotation}). The second step provides an instance and a category level segmentation on objects in each region. Given a 3D mesh of region (Fig. \ref{fig:matterport_object_annotation_1}), the first type of segmentation (instance-level) assigns a different label for each object instance (Fig. \ref{fig:matterport_object_annotation_2}) while the latter (category-level) associates different labels for different object types (Fig. \ref{fig:matterport_object_annotation_3}). To do that, the authors extract a mesh for each region and process it with the crowd-source interface of ScanNet  proposed by \citeauthor{scannet} \cite{scannet}. Using the open-source code provided with ScanNet, the authors of Matterport3D perform surface reconstruction of regions' meshes to ``paint'' triangles to segment and name all object instances
within each region. The 3D segmentations contain a total of 50,811 object
instance annotations divided into 40 objects categories.

\begin{figure}[h!]
	\centering
	\begin{subfigure}[b]{\linewidth}
		\centering
		\begin{subfigure}[b]{0.48\linewidth}
			\centering
			\includegraphics[width=\textwidth]{images/matterport_surfaces_by_label.pdf}
		\end{subfigure}
		\hfil
		\begin{subfigure}[b]{0.48\linewidth}
			\centering
			\includegraphics[width=\textwidth]{images/surfaces_by_label.pdf}
			
		\end{subfigure}
	\caption{}
	\label{fig:matterport-floor-annotation}
	\end{subfigure}
	\newline
	\begin{subfigure}[b]{\linewidth}
		\centering
		\begin{subfigure}[b]{0.32\linewidth}
			\centering
			\includegraphics[width=\textwidth]{images/matterport_room22_color.pdf}
			\caption{}
			\label{fig:matterport_object_annotation_1}
		\end{subfigure}
		\hfil
		\begin{subfigure}[b]{0.32\linewidth}
			\centering
			\includegraphics[width=\textwidth]{images/matterport_room22_instances.pdf}
			\caption{}
			\label{fig:matterport_object_annotation_2}
		\end{subfigure}
		\hfil
		\begin{subfigure}[b]{0.32\linewidth}
			\centering
			\includegraphics[width=\textwidth]{images/matterport_room22_categories.pdf}
			\caption{}
			\label{fig:matterport_object_annotation_3}
		\end{subfigure}
	\end{subfigure}
	\caption{(a) Region annotation on floor plans. (b, c, d) Instance and category object segmentation of floor plan's region. Images from \cite{matterport}.}
\end{figure}

\subsection{New Gibson Version}
\label{sec:new_gibson_version}

Gibson and Matterport3D are the selected packages to acquire the dataset to train and evaluate the proposed robotic doors detector. In the first experiments, we argue that these technologies present some problems and limitations that affect an easy and fast data collection procedure. 

A suitable way for acquiring a visual robotic dataset in simulation is to embodies a virtual agent which autonomously explores a simulated environment using the standard navigation stack. In this way, the examples captured during its navigation are coherent with a real exploration strategy, reflecting how a real robot perceives an indoor scene. Unfortunately, this solution is infeasible with the selected technologies. 

First of all, this technique is extremely time-consuming. Gibson perfectly simulates the real features of mobile agents (e.g. the speed of movement, the sensors' accuracy, and the physical characteristics). This fact allows us to use the standard navigation stack that is generally provided for well-known mobile robots (e.g. Turtlebot2 \cite{turtlebot2}, Turtlebot3 \cite{turtlebot3}, or Husky car \cite{husky}). Despite this, the real-time simulation provided by Gibson makes the data acquisition extremely slow, especially for large environments.  Furthermore, the dataset would be suitable for multiple robots with different navigation packages and camera heights. This fact implies performing multiple runs in the same environments with different agents, further extending the acquisition time. Due to navigation is a challenging task, an autonomous agent can crash into an obstacle or iterate the same actions due to a software issue. In simulated environments, fixing robot failures is often infeasible, making the runs useless. 

Another issue is related to the exploration, which can be performed through different strategies (such as frontier-based exploration \cite{frontierexploration}) that do not guarantee the experiments' repeatability. In different runs, the robot can follow different paths and the exploration time can vary a lot. In this way, the dataset is heavily dependent on the chosen exploration strategy.

While the issues just reported can be solved using appropriate techniques, the problem related to the Matterport3D dataset makes completely infeasible data collection using Gibson ``as is''. The physical structure of the Matterport's environments is encoded in 3D polygonal meshes. A polygon mesh is a collection of vertices, edges, and faces that defines the shape of a polyhedral object. These meshes are extremely cluttered and noisy. The furniture models are malformed and incomplete, e.g. a table can be composed of the horizontal plan omitting one or more legs (Fig. \ref{fig:matterport_issues_meshes_furniture}), or some faces are disconnected from the principal object mesh. Furthermore, the walls often present holes at windows, mirrors, or other undefined locations (Fig. \ref{fig:matterport_issues_hole}). These artifacts affect the robot perception, making autonomous navigation impossible. Another important aspect concerns the floor plans' surfaces, that are extremely irregular. They are composed of a series of triangles but the vertexes are not perfectly aligned (Fig. \ref{fig:matterport_issues_floor_plan}). This further complicates the robot navigation, which often crashes over floor irregularities.

\begin{figure}[h!]
	\centering
	\begin{subfigure}[b]{0.32\linewidth}
		\centering
		\includegraphics[width=\textwidth]{images/table.png}
		\caption{}
		\label{fig:matterport_issues_meshes_furniture}
	\end{subfigure}
	\hfil
	\begin{subfigure}[b]{0.32\linewidth}
		\centering
		\includegraphics[width=\textwidth]{images/holes.png}
		\caption{}
		\label{fig:matterport_issues_hole}
	\end{subfigure}
	\hfil
	\begin{subfigure}[b]{0.32\linewidth}
		\centering
		\includegraphics[width=\textwidth]{images/floor_plan_irregularity.png}
		\caption{}
		\label{fig:matterport_issues_floor_plan}
	\end{subfigure}
	\caption{Matterport3D mesh malformations. (a) A suspended table and chair. (b) A holes in a bedroom. (c) Floor plan irregularities.}
\end{figure}


To overcome these difficulties, we provide an upgraded version of Gibson available on Github\footnote{The code of the new Gibson version: \url{https://github.com/micheleantonazzi/GibsonEnv}.}. We developed a new simulation mechanism without any physical constraints: gravity and collision detection are removed. Inside this stage, the robot can not move autonomously using motors, but the user can set an absolute position and orientation in which the robot must be located at each instant. In other words, we implement a less realistic simulation mechanism in which a robot can both instantly teleport in any location and traverse obstacles. In this way, the user can perform data acquisition in batch without any failure. The robot does not crash over the floor's irregularities (that are due to artifacts), does not go out of the mesh, and can autonomously traverse multiple floors without dealing with architectural barriers (such as stairs or elevators).

The new version of Gibson we released implements further improvements. First of all, we resolve some building errors (that occurs especially in Ubuntu 20.04), and automatize the compile procedure, implementing it inside the \textsf{setup.py} file. In this way, with the simple command \textsf{pip install gibson}, the simulation environment is compiled and installed. The necessary dependencies have been added as Git sub-modules. In this way, they are automatically downloaded and compiled, simplifying the dependencies management. Then, we developed a new and more efficient module to manage Gibson's assets, such as the neural network's weights, the virtualized agents, or the worlds dataset. We also offer a command-line interface to automatically download and extract the assets files from the links provided by the authors. To use it, the user can type three different commands (after Gibson's installation) in a console terminal: \textsf{gibson-set-assets-path} for specifying the assets' folder, \textsf{gibson-download-dataset} to download the provided worlds dataset, and \textsf{gibson-download-assets-core} to retrieve the remaining assets files. Finally, we provide a compiled version of Gibson available on PyPI\footnote{The compiled version of Gibson: \url{https://pypi.org/project/gibson/}.} following the manylinux standard\footnote{The manylinux repository: \url{https://github.com/pypa/manylinux}.}. We offers also a utility package, called \textit{gibson-env-utilities}\footnote{The gibson-env-utilities source code: \url{https://github.com/micheleantonazzi/gibson-env-utilities}.}, to facilitate the usage of Gibson. It automatizes the launch of simulation runs, by offering a class that automatically creates the Gibson configuration file in which the parameters of the run are specified. Furthermore, \textit{gibson-env-utilities} provides a mechanism to store and retrieve the metadata related to Gibson's scenes. This metadata includes the environment name, the belonging dataset (Matterport3D \cite{matterport} or Stanford-2D-3D \cite{stanford2d3d}) a boolean value that indicates if the environment is semantically annotated, and the number of floors. In addition, for each floor is provided the floor's height and a valid position on such a floor in which the robot can be placed.


\section{Pose Estimator}
\label{sec:pose_estimator}
Thanks to our new simulation stage integrated into the new version of Gibson (described in Sec. \ref{sec:new_gibson_version}), a virtual autonomous agent can be freely positioned and oriented in any environment location. Despite this, to acquire a vision dataset, we need to determine the valid positions in which the robot can be placed: it must be in free space inside the building perimeter, not overlapping obstacles (walls, floors, or furniture). Furthermore, the dataset is used in a robotic vision application, so the samples must be collected in locations consistent with a possible exploration strategy followed by a mobile agent. Typically, a robot walks away from obstacles and chooses the shortest path for moving between different areas. To address these requirements, we developed a method called Pose Estimator, which is based on the work proposed by \citeauthor{repeatabilityslamarxiv} in \cite{repeatabilityslamarxiv, repeatabilityslam}.

Pose Estimator is the component that extracts plausible positions for an autonomous agent from which collect the dataset. This tool performs the computation of the Voronoi graph over the 2D occupancy grid map \cite{cuupancygridfirst} of a floor plan and chooses the positions by sub-sampling this graph. A Voronoi diagram is a partition of a plane into regions close to a given set of points (also called seeds, sites, or generators). Each Voronoi cell contains all points of the plane closer to that seed than to any others. The following paragraphs describe the procedures performed by this module, which is implemented in the \textit{gibson-env-utilities} package.



\paragraph{Extract the 2D Occupancy Grid} The first step concerns the computation of a 2D occupancy grid map (Fig. \ref{fig:pose_estimator_occupancy_grid}) of a floor starting from a 3D mesh (Fig. \ref{fig:pose_estimator_3dmesh}). Each environment of Matterport3D is modeled by a 3D Wavefront file (with \textit{.obj} extension) that stores the position of each vertex and the faces that compose each polygon. The 3D model of an environment is rendered, using a software like Blender\footnote{The Blender's web page: \url{https://www.blender.org/}.}, to estimate the average height of the points that compose the selected floor plan. Then, the method performs multiple cross-sections of the 3D mesh with parallel planes, starting from a few centimeters over the floor. The 2D cross-sections, that detect obstacles at multiple heights, are aggregated and projected in a 2D image. Finally, this image is manually fixed to fill the shortcomings due to mesh inaccuracies or to remove artifacts produced by the cross-section. In addition, it is particularly important that the user closes all the opened contours in the image. This final 2D image (Fig. \ref{fig:pose_estimator_occupancy_grid}) stores the occupancy grid map of a floor plan, where each pixel represents a sub-portion of the environment and contains the probability that it is occupied by an obstacle. Together with the image map, important metadata are also saved, in order to enable conversion operations between map and simulation coordinates. The metadata includes the map's origin (in pixels) and the scale, which specifies how many meters correspond to a pixel.

These operations are implemented in the \textsf{GibsonAssetsUtilities} class of \textit{gibson-env-utilities} package. In particular, \textsf{load\_obj} method loads the Wavefront file while \textsf{create\_floor\_map} extract the floor map and the relative metadata, saving them to the disk. The mesh file loading and the computation of multiple cross-sections are performed using Trimesh\footnote{The Trimesh source code: \url{https://github.com/mikedh/trimesh}.}, a pure Python library for loading and elaborating triangular meshes. The final image is rendered using Matplotlib\footnote{The Matplotlib web site: \url{https://matplotlib.org/}.}, a comprehensive library for creating static, animated, and interactive visualizations in Python. 

\begin{figure}[h!]
	\centering
	\begin{subfigure}[b]{0.49\linewidth}
		\centering
		\includegraphics[width=\textwidth]{images/pose_estimator_3Dmesh.png}
		\caption{}
		\label{fig:pose_estimator_3dmesh}
	\end{subfigure}
	\hfil
	\begin{subfigure}[b]{0.49\linewidth}
		\centering
		\includegraphics[width=\textwidth]{images/pose_estimator_1.png}
		\caption{}
		\label{fig:pose_estimator_occupancy_grid}
	\end{subfigure}
	\caption{The computation of the 2D occupancy grid map (right) from the 3D mesh (left) of an environment.}
\end{figure}

\newpage

\paragraph{Compute the Voronoi Bitmap} This step aims to extract the Voronoi bitmap graph from the 2D occupancy grid of a floor (Fig. \ref{fig:pose_estimator_1}). In order to do this, Pose Estimator processes the occupancy grid map, stored in a \textsf{png} image, using OpenCV\footnote{The OpenCV home page \url{https://opencv.org/}.}: a library of programming functions mainly aimed at real-time computer vision. At first, the image is thresholded to obtain a more uniform mask which indicates the free and occupied location. To do this, the module uses the OpenCV's \textsf{threshold} method, which suppresses (leads to zero) all pixels with a color value less than 250. Then, this optimized occupancy grid is eroded and dilated (through the \textsf{erode} and \textsf{dilate} methods of OpenCV) using a $3 \times 3$ kernel to reduce the noise. Now, the method finds the contours (using the OpenCV \textsf{findContours} method) in the smoothed image to identify the boundaries in which the robot can move. The methods return all the contours in the image without any approximation organized in a tree structure, thanks to the \textsf{CHAIN\_APPROX\_NONE} and \textsf{RETR\_TREE} flags respectively. The longest contour is assumed to be the floor's perimeter. The area outside the floor's outline and all its internal contours (that represent the furniture) are black-filled: the resulting image (Fig. \ref{fig:pose_estimator_filled}) highlights in white the free area in which the robot can move. Now, the Voronoi diagram is calculated using the Delaunay triangulation \cite{delaunayproof} (implemented by the OpenCV class \textsf{SubDiv2D}) using all the points belonging to the identified contours.  The Voronoi facets, obtained with \textsf{getVoronoiFacetList} method of the \textsf{SubDiv2D} instance, are drawn in a new image, reporting only the segments inside the white area of the black filled image. The drawn segments compose the bitmap of the Voronoi graph (Fig. \ref{fig:pose_estimator_voronoi_bitmap}) computed over a floor occupancy grid map. As shown by (Fig. \ref{fig:pose_estimator_voronoi_bitmap_map}), the Voronoi bitmap does not exceed the floor's contours and does not overlap furniture. 

\begin{figure}[h!]
	\centering
	\begin{subfigure}[b]{0.49\linewidth}
		\centering
		\includegraphics[width=\textwidth]{images/pose_estimator_1.png}
		\caption{}
		\label{fig:pose_estimator_1}
	\end{subfigure}
	\hfil
	\begin{subfigure}[b]{0.49\linewidth}
		\centering
		\includegraphics[width=\textwidth]{images/pose_estimator_filled_contours.png}
		\caption{}
		\label{fig:pose_estimator_filled}
	\end{subfigure}
	\newline
	\begin{subfigure}[b]{0.49\linewidth}
		\centering
		\includegraphics[width=\textwidth]{images/pose_estimator_bitmap_voronoi.png}
		\caption{}
		\label{fig:pose_estimator_voronoi_bitmap}
	\end{subfigure}
	\hfil
	\begin{subfigure}[b]{0.49\linewidth}
		\centering
		\includegraphics[width=\textwidth]{images/pose_estimator_bitmap_map.png}
		\caption{}
		\label{fig:pose_estimator_voronoi_bitmap_map}
	\end{subfigure}
	\caption{The computation of the Voronoi bitmap. In a 2D occupancy grid map of a floor (a), the free space is isolated in white (b), and the Voronoi graph is generated through Delaunay triangulation (c, d).}
	\label{fig:pose_estimator}
\end{figure}

\newpage

\paragraph{Voronoi Graph Pruning and Digitization} The last step aims to extracts the graph described by the Voronoi bitmap and prune it to remove the spurious sidelines. More formally, we represent this graph as $G = (V, E)$, where the vertices are the black pixels in the Voronoi bitmap and the edges (that define symmetric relations between vertices) are implicitly stored inside nodes (each nodes specify its neighbors). 

To build a graph starting from a Voronoi bitmap (Fig. \ref{fig:pose_estimator_bitmap}), we parse the image to build an instance node for each black pixel in the bitmap, adding them to a graph object. The \textsf{node} and \textsf{graph} classes are implemented in a file called \textsf{utilities/graph.py} of \textit{gibson-env-utilities} package. Each node is described by its coordinates in the image expressed in pixel. Then, the method proceeds with the identification of adjacent nodes that compose the graph. For each node, the approach searches the neighboring nodes inside the $3 \times 3$ pixels area centered in the current node's image coordinates. If a pixel in such area belongs to the Voronoi bitmap, the correspondent node is added to the neighbors' list of the current node and vice-versa, creating a double connection between them. The graph can be composed of multiple sub-components disconnected from each other. In other words, the initial graph $G = (N, V)$ is partitioned in multiple sub-graphs $g_1, g_2, ..., g_n$, where each sub-graphs $g_i$ is a connected component formed by a sub-set of nodes $g_i \in N \setminus \bigcup_{y=1}^{n} g_{y}, \mid y  \neq i$. Initially, the method preserves all the sub-graphs. 

The algorithm proceeds pruning the spurious sidelines. To do this, the method parses each sub-graph to recursively remove each node with a single neighbor. In other words, when the recursive function encounters the end of a sideline, it deletes the last node and consequently removes the others until the line beginning, where there is a crossroads. A plot of the pruned graphs is shown in Fig. \ref{fig:pose_estimator_pruned_lines}. 

Then the graph is printed in a new Voronoi bitmap to be further refined. The new image is subject to a dilation operation immediately followed by an image skeletonization. These two operations are necessary to obtain a bitmap Voronoi graph as clean and uniform as possible. The images dilation is performed through the OpenCV's \textsf{dilate} method, applied with a $3 \times 3$ kernel, while the skeletonization is implemented by a function provided by scikit-image\footnote{The scikit-image website: \url{https://scikit-image.org/}.}, a library designed for image processing.
Since the skeletonization function requires a binary image expressed in $[0, 1]$ range, the Voronoi bitmap is converted in a binary image encoded in such interval. After this operation, the image is re-converted in the OpenCV's format, in which pixels can assume a value between 0 and 255. The skeletonized Voronoi bitmap is shown in Fig. \ref{fig:pose_estimator_skelethon}. 

Finally, the last pruned and skeletonized Voronoi bitmap is re-parsed to build a new and more refined Voronoi graph, using the same procedure previously explained. A representation of the final graph can be seen in Fig. \ref{fig:pose_estimator_skelethon_map}. 
 
\begin{figure}[h!]
	\centering
	\begin{subfigure}[b]{0.49\linewidth}
		\centering
		\includegraphics[width=\textwidth]{images/pose_estimator_bitmap_voronoi.png}
		\caption{}
		\label{fig:pose_estimator_bitmap}
	\end{subfigure}
	\hfil
	\begin{subfigure}[b]{0.49\linewidth}
		\centering
		\includegraphics[width=\textwidth]{images/pose_estimator_pruned_lines.png}
		\caption{}
		\label{fig:pose_estimator_pruned_lines}
	\end{subfigure}
	\newline
	\begin{subfigure}[b]{0.49\linewidth}
		\centering
		\includegraphics[width=\textwidth]{images/pose_estimator_skelethonization.png}
		\caption{}
		\label{fig:pose_estimator_skelethon}
	\end{subfigure}
	\hfil
	\begin{subfigure}[b]{0.49\linewidth}
		\centering
		\includegraphics[width=\textwidth]{images/pose_estimator_skelethon_map.png}
		\caption{}
		\label{fig:pose_estimator_skelethon_map}
	\end{subfigure}
	\caption{The steps for pruning the Voronoi bitmap. The initial image (a) is digitized and pruned using a recursive function to remove the side lines (b). Then, the Voronoi graph is printed in a new image and skeletonized using scikit-image function (c, d). }
\end{figure}

\paragraph{Voronoi Graph Sub-Sampling} The last step performed by Pose Estimator concerns sub-sampling the Voronoi graph to extract the positions from which collect the visual dataset. As mentioned in the previous paragraph, the Voronoi graph is composed of multiple sub-graphs, that are not connected with each other. 

At first, the method discards all the sub-graphs except the longest one. Then, the approach extracts from the Voronoi graph a list of positions divided by a certain distance. The distance value is passed as a hyper-parameter, called \textsf{interval}, that controls the number and the granularity of the extracted location. The Voronoi graph is composed of multiple nodes that define points in the simulated environment. This procedure starts from an initial node described with its coordinates in the Voronoi bitmap and calculates its position in the simulated environment using the map's metadata (the map's origin and the scale). After that, the method begins to walk through the graph until a certain distance has been reached: at this point a position is sub-sampled. The method runs until the entire graph has been traversed, returning a list of positions equally spaced in the Voronoi graph (as shown by Fig. \ref{fig:pose_estimator_subsampled}). The distance between two adjacent nodes is computed by calculating the length of the legs of the right triangle between two bridges and applying the Pythagorean theorem to find the hypotenuse. Typically, the selected positions are located along the edge that connects two adjacent vertices. The conversions between the Voronoi bitmap coordinates (expressed in pixels) and the simulated coordinates (in meters) are performed by the \textsf{Coordinates} class, coded inside the \textsf{utilities/graph.py} file of \textit{gibson-env-utilities} package.

\begin{figure}[h!]
	\centering
	\begin{subfigure}[b]{0.49\linewidth}
		\centering
		\includegraphics[width=\textwidth]{images/pose_estimator_skelethon_map.png}
		\caption{}
		\label{fig:pose_estimator_voronoi_graph}
	\end{subfigure}
	\hfil
	\begin{subfigure}[b]{0.49\linewidth}
		\centering
		\includegraphics[width=\textwidth]{images/pose_estimator_subsampled.png}
		\caption{}
		\label{fig:pose_estimator_subsampled}
	\end{subfigure}
	\caption{The sub-sampling of the Voronoi graph.}
\end{figure}

\section{Dataset}

This thesis proposes a module to detect doors in autonomous mobile robots. Our detector, as described in Sec. \ref{sec:doors_detector}, is based on DETR, a deep end-to-end module based on Transformers \cite{transformer} that performs object detection in RGB images. In a deep learning application, the dataset plays a crucial role to determine the final results. 

The virtualization environment used in this thesis (the modified version of Gibson described in Sec. \ref{sec:new_gibson_version}) offers three types of data during a simulation run: RGB images, depth data, and semantic information. The RGB images are tri-dimensional vectors $W \times H \times 3$, where $W = H$ and the last dimension contains the pixels' colors (red, green and, blue). Depth data are encoded in bi-dimensional array with the same dimensions as the RGB images. In this way, a depth value is assigned to each pixel. Finally, semantic information is furnished in other RGB images, where the pixels are colored according to the object categories they belong to. Gibson provides 40 object categories (including doors), tagged with a numeric code. In the semantic images, the numerical code of each category is converted in an RGB color with the following procedure:

\begin{equation}
\label{eq:pareto mle2}
\begin{aligned}
B &= (\text{ID}) &\mod 256, \\
G &= (\text{ID} \gg 8) &\mod 256, \\
R &= (\text{ID} \gg 16) &\mod 256,
\end{aligned}
\end{equation}
where ID is the object category code and $\gg$ represents the bitwise right shift operation. 

In this section, we present the framework we develop to manage the vision dataset collected in this thesis. Then, we proceed by describing the dataset labeling procedure, reporting the difficulties encountered during this phase, and defining the final dataset's composition.

\subsection{The Dataset Management Framework}
\label{sec:generic_dataset}
In the initial phase of this thesis, we did not know how the final visual dataset was composed and what data characterized the collected examples. To overcome this uncertainty, we develop \textit{generic-dataset}\footnote{The generic-dataset repo: \url{https://github.com/micheleantonazzi/generic-dataset}.}, a configurable framework that automatically generates the code and the necessary classes to manage a dataset of any kind. This is possible using the \textit{meta-programming} paradigm offered by Python. Meta-programming is a technique in which computer programs have the ability to generate new code, create other programs, and modify their internal structure while running. This allows programs greater flexibility to efficiently handle new situations without recompilation. 

In Python, the meta-programming paradigm is implemented using \textit{decorators} and \textit{meta-classes}. A decorator allows programmers to modify the behavior of functions or classes. In other words, a decorator wraps an entity into a function in order to extend the behavior of the wrapped function or class, without permanently modifying it. The meta-classes, otherwise, represent a further implementation of meta-programming. In Python, everything is an object and classes are objects as well. A class in Python must have a type and it is an instance of another super-type, called meta-class. In simple terms, a meta-class is the definition of a class. The default meta-class which is responsible for making classes is called \textit{type}. Fig. \ref{fig:metaclass} visually explains the concept just reported: an object is an instance of a class and a class is an instance of a meta-class.

\begin{figure}[h!]
	\centering
	\includegraphics[width=\textwidth]{images/metaclass-hierarchy.jpeg}
	\caption{The hierarchy of objects, classes and meta-classes.}
	\label{fig:metaclass}
\end{figure}

Thanks to meta-programming, the programmer can write its custom meta-classes to modify the way from which classes are generated by performing extra actions or injecting code. By exploiting this principle, the \textit{generic-dataset} framework offers an intuitive API that creates a custom class to model a particular examples' type of a generic dataset. To do this, this API, implemented by a \textsf{SampleGenerator} object, creates a meta-class according to the directions of the programmer, that defines the final desired class to deal with precise dataset's examples. In the constructor of \textsf{SampleGenerator}, the user specifies the name of the generated sample class and the label set. In this way, both regression or classification problems can be modeled. In a regression problem, the labels are real numbers and the label space is typically infinite. In this first case, the label set passed to the constructor must be empty. Otherwise, in a classification problem, the label set must contain all the possible labels that an example can assume. The programmer can also add data fields to the final generated class, specifying their name and type. In this case, the framework automatically generates useful methods to manipulate the custom fields (like getters and setters functions). Furthermore, the user can add custom methods to the final class in order to manipulate the example's fields. Finally, using the \textsf{generate\_sample\_class} method, \textsf{SampleGenerator}, the user obtains the generated custom class (which is an instance of the configured meta-class) which models a specific type of example. The instances of the generated class, that deal directly with the examples, are thread-safe. The generated class automatically implements this feature by assigning a lock to each data field. Then, each method is decorated with a function that acquires the locks of the fields used within each method before execution and then releases them. Also, the custom methods can be synchronized by specifying the name of the used fields.

Despite we did not know the exact final composition of our dataset, surely we will deal with image manipulation. To address this requirement,  \textit{generic-dataset} offers a useful utility to manipulate RGB images, that in Python are commonly stored in NumPy arrays. This utility, implemented by the \textsf{DataPipeline} class, generates an elaboration pipeline to modify a NumPy array. A pipeline consists of a series of operations performed consecutively that can be executed in CPU or in GPU according to the programmer's needs without changing the code. This is possible using CuPy\footnote{The CuPy web page: \url{https://cupy.dev/}.}, an open-source array library for GPU-accelerated computing with Python. The CuPy's interface is highly compatible with NumPy, allowing to write agnostic programs which can be executed in CPU or GPU by replacing the engine (NumPy or CuPy), without any code change. A pipeline can be customized by adding functions to modify the initial array and then executed using the \textsf{run} method. If the \textsf{use\_gpu} flag is set as \textsf{False}, the pipeline is synchronously executed in the CPU. Otherwise, if such flag is \textsf{True}, the operations are asynchronously performed by the GPU, so the user must synchronize the two elaboration units through the \textsf{get\_data} method. This mechanism is integrated into the class generated by \textsf{SampleGenerator}. It offers the possibility to automatically create an elaboration pipeline for the fields of the generated sample class. In addition, pipelines are protected with a dedicated lock which prevents data access and modification when during the execution of the correspondent pipeline.

The \textit{generic\_dataset} framework provides a mechanism to manage the dataset's persistence. It automatically organizes the folder hierarchy to store and organize the dataset and offers the necessary methods to save and load the examples. The classes generated by \textsf{SampleGenerator} are sub-type of \textsf{GenericSample} class, which provides a utility method to manage an example instance class of any kind. When the programmer adds a new field to the generated class, it must specify if it belongs to the dataset and, if so, it must provide the necessary functions for saving and loading such data type to the disk. The dataset folder hierarchy is organized as follows. The main directory is divided into sub-folders, that could specify different data categories or different moments in which the data are collected. Then, for a classification problem, the samples are divided into another level of sub-directories according to their label. Otherwise, in a regression task, the samples are saved in the same directory and the labels are stored in a dedicated file. Finally, the examples' fields are saved in different folders and the files inside them are named to reconstruct the acquisition order. More precisely, the file name contains two counters: the relative count and the absolute count. The first indicates the example's number based in its label's folder while the latter is the absolute count of the sample over all dataset. For a regression problem, these two values are equal because all examples belong to the same directory. Fig. \ref{fig:structuredataset} visually explain the folder hierarchy of a classification (Fig. \ref{fig:structureclassification}) and a regression (Fig. \ref{fig:structureregression}) problem. The entire dataset can be managed by an instance of \textsf{DatasetManager} class, while each folder at the first nesting level is controlled by an instance of \textsf{DatasetFolderManager}.

\begin{figure}[h!]
	\centering
	\begin{subfigure}[b]{0.5\textwidth}
		\dirtree{%
			.1 MAIN\_DATASET\_FOLDER.
			.2 FOLDER\_1.
			.3 0.
			.4 FIELD\_1.
			.5 field\_name\_rc\_ac.
			.5 field\_name\_rc\_ac.
			.5 \vdots.
			.4 FIELD\_2.
			.5 \vdots.
			.3 1.
			.4 FIELD\_1.
			.5 \vdots.
			.4 FIELD\_2.
			.5 \vdots.
			.2 FOLDER\_2.
			.3 \vdots.
			.2 \vdots.
		}
		\caption{}
		\label{fig:structureclassification}
	\end{subfigure}
	\begin{subfigure}[b]{0.4\textwidth}
		\dirtree{%
			.1 MAIN\_DATASET\_FOLDER.
			.2 FOLDER\_1.
			.3 LABEL.
			.4 label\_rc\_ac.
			.4 label\_rc\_ac.
			.4 \vdots.
			.3 FIELD\_1.
			.4 field\_name\_rc\_ac.
			.4 field\_name\_rc\_ac.
			.4 \vdots.
			.3 FIELD\_2.
			.4 field\_name\_rc\_ac.
			.4 field\_name\_rc\_ac.
			.4 \vdots.
			.2 FOLDER\_2.
			.3 \vdots.
			.2 \vdots.
		}
		\caption{}
		\label{fig:structureregression}
	\end{subfigure}
	\caption{The structure of a binary classification dataset (a) and a regression dataset (b). The files' names are are lowercase, where \textsf{rc} indicates the relative count of an example inside its label's folder, and \textsf{ac} represents the example's absolute count. }
	\label{fig:structuredataset}
\end{figure}


\subsection{Dataset Labeling and Composition}
\label{sec:dataset_labeling}
The labeling procedure in our robotic object detection task consists of dividing the positive examples (that contain the object of interest) from the negative ones and identifying the bounding boxes around the objects' instances. A vision dataset is mainly composed of RGB images, but it must specify the coordinates of the bounding boxes and their object categories in each image.

In our first dataset definition, each example is composed of all the data provided by Gibson: an RGB image, the semantic data, and the relative semantic information. The bounding boxes coordinates are not stored in dedicated files, but they are automatically calculated using  the semantic data provided by Gibson. To do this, we developed a dedicated function through the OpenCV's methods. At first, a semantic image is binarized assigning a precise color to the pixels that belong to a door and suppressing the others. Then, this function founds the contours in the binary image using the OpenCV's \textsf{findContours} method. Finally, a bounding box is created (through the \textsf{boundingRect} method of OpenCV) for each detected contour that contains a door. This function is useful because automatically designs the bounding boxes using semantic data.

Despite the usefulness of this method, we were forced to discard it after a few experiments. The main problems regard the semantic annotations. At first,  the knowledge provided by the semantic data is are not enough informative to automatically label the dataset used in this thesis. This is because, semantic information does not specify the doors' status (open or closed) and does not include implicit doors, e.g. wall openings as described in Sec. \ref{sec:door_definition}. Furthermore, the Matterport3D scenes are categorized in an inaccurate and noisy way. The tagging procedure is extremely error-prone because the semantic information is provided by labeling the vertexes of the 3D environments' meshes. Specifically identifying objects in hundreds of thousands of vertices certainly leads to errors. Examining some collected examples, we argue that some doors end up on the wall (Fig. \ref{fig:wrong_box_1}) or are partially tagged. Furthermore, adjacent doors are difficult to distinguish and often are not clearly separated in the semantic image. Another wrong situation happens when the robot does not frame the upper door jamb. In this case, the two lateral sites are recognized as different doors because they do not share pixels in the semantic image (Fig. \ref{fig:wrong_box_2}). Due to these issues, automatically finding the bounding boxes through semantic annotation is unfeasible. As shown in Fig. \ref{fig:wrong_box}, the automatic labeling procedure using only the semantic data introduces noise that can degrade the doors detector's accuracy.

\begin{figure}[h!]
	\centering
	\begin{subfigure}[b]{\linewidth}
		\centering
		\begin{subfigure}[b]{0.32\linewidth}
			\includegraphics[width=\textwidth]{images/wrong_box_rgb_1.png}
		\end{subfigure}
		\hfil
		\begin{subfigure}[b]{0.32\linewidth}
			\includegraphics[width=\textwidth]{images/wrong_box_semantic_1.png}
		\end{subfigure}
		\hfil
		\begin{subfigure}[b]{0.32\linewidth}
			\includegraphics[width=\textwidth]{images/wrong_box_box_1.png}
		\end{subfigure}
		
		\caption{}
		\label{fig:wrong_box_1}
	\end{subfigure}
	\newline
	\begin{subfigure}[b]{\linewidth}
		\centering
		\begin{subfigure}[b]{0.32\linewidth}
			\includegraphics[width=\textwidth]{images/wrong_box_rgb_2.png}
		\end{subfigure}
		\hfil
		\begin{subfigure}[b]{0.32\linewidth}
			\includegraphics[width=\textwidth]{images/wrong_box_semantic_2.png}
		\end{subfigure}
		\hfil
		\begin{subfigure}[b]{0.32\linewidth}
			\includegraphics[width=\textwidth]{images/wrong_box_box_2.png}
		\end{subfigure}
		\caption{}
		\label{fig:wrong_box_2}
	\end{subfigure}

	\caption{Examples with inaccurate semantic annotations. Each row report two examples of wrong bounding boxes obtained through semantic data. From left to right, the figure shows the RGB image, the relative semantic data, and the bounding boxes derived from semantic information for each example.}
	\label{fig:wrong_box}
\end{figure}

The dataset acquisition has been performed using a hybrid method, which includes automatized procedures and the intervention of a human operator. At first, we acquired the dataset by saving all the data provided by Gibson (RGB and semantic images, and depth data). As mentioned in Sec. \ref{sec:solution}, a visual robotic dataset must contain also negative samples, so the first step concerns the discrimination between negative and positive data points. Despite the inaccuracy of the semantic data, we used them to separate the examples according to the door's presence: if the semantic image does not have pixel related to a door, the relative example is tagged as negative, as positive otherwise. Then, a human operator parses all the positive examples to define the bounding boxes and the related door's status (open or closed). For this purpose, we developed an intuitive visual tool inside the \textsf{scripts/data\_annotator.py} file of the \textit{gibson-env-utilities} package. This program loads each positive example and displays the bounding boxes extracted using the semantic data. The user can fix them by deleting the wrong bounding boxes and creating new ones, specifying also the doors' status. Some positive examples can not be used for training the doors detector module, e.g. if the robot is too close or too far to a door. In the first case, the RGB image depicts a uniform color not exposing the typical door's feature, likewise, in the second case, the doors appear as a small uniform rectangular. In particular, we discarded a door too close to the acquisition position considering the depth data: if the average distance is less that $0.30 m$, the bounding box is not displayed. Likewise, the doors too far with respect to the robot position are not considered if they cover less than $2,5\%$ of the total semantic image. In such situations, the tool we developed does not displays the bounding boxes. In this way, the user understands which doors should not be considered. In addition, the examples with no valid doors are discarded. The final doors dataset is composed of negative and positive examples. Each example is characterized by an RBG image, the depth data, and an array with contains the bounding boxes' coordinates and the related status. In the negative examples, the list of the bounding boxes is empty.

\begin{figure}[h!]
	\centering
	\begin{subfigure}[b]{\linewidth}
		\centering
		\begin{subfigure}[b]{0.4\linewidth}
			\includegraphics[width=\textwidth]{images/correct_box_rgb_1.png}
		\end{subfigure}
		\hfil
		\begin{subfigure}[b]{0.4\linewidth}
			\includegraphics[width=\textwidth]{images/correct_box_box_1.png}
		\end{subfigure}
		\caption{}
		\label{fig:correct_box_1}
	\end{subfigure}
	\newline
	\begin{subfigure}[b]{\linewidth}
		\centering
		\begin{subfigure}[b]{0.4\linewidth}
			\includegraphics[width=\textwidth]{images/correct_box_rgb_2.png}
		\end{subfigure}
		\hfil
		\begin{subfigure}[b]{0.4\linewidth}
			\includegraphics[width=\textwidth]{images/correct_box_box_2.png}
		\end{subfigure}
	\caption{}	
	\label{fig:correct_box_2}
	\end{subfigure}

	\caption{The fixed examples of Fig. \ref{fig:wrong_box}. These two data point belong to the final dataset labeled by a human operator.}
	\label{fig:correct_box}
\end{figure}

\section{Model Evaluator}
\label{sec:model_evaluator}
This module is responsible to evaluate the performance of the door's detector trained with the collected dataset. As mentioned in Sec. \ref{sec:goals}, we collect a dataset suitable for a robotic vision application, so it is composed of positive and negative examples (where the firsts contain open or closed doors while the latter do not contain any objects of interest). The metric we implement considers these two types of examples to better evaluate the model if used by an autonomous agent. 

The evaluation method proposed by this thesis is based on the metric of Pascal VOC challenge \cite{pascal}. This metric solves several issues related to the evaluation of models that performs object detection or segmentation. In an object detection task, images contain multiple object categories or multiple instances of the same object, so a standard approach to determine which one of the $m$ classes an image contains and where it is can not be used. Furthermore, the prior distribution over different classes is significantly nonuniform so a
simple accuracy measure (percentage of correctly classified
examples) is not appropriate. During the evaluation, it is also necessary
to evaluate the trade-off between different types of classification error (e.g. false negative or false positive). The metric of the Pascal VOC challenge computes a separate ``score'' over each class to evaluate the model's performance in detecting each object category.  This metric calculates the interpolated Average Precision (AP), proposed by \citeauthor{averageprecision} in \cite{averageprecision}, over all classes in the positive images, that are closed and open door in the context of this thesis. AP is determined by the area under the precision/recall curve in the $[0, 1]$ interval. Precision and recall are two measures that differently relate the true positive (TP), false positive (FP), and false negative (FN). In an object detection task where the predictions are bounding boxes, TPs are objects correctly recognized, FNs are objects not detected, while FPs are bounding boxes that do not correspond to any object. Precision measures how accurate the predictions are, i.e. the percentage of the correct predictions over the total number of objects to detect.

It requires as input the predicted bounding boxes with their confidence score.

\begin{equation}
\label{eq:precision}
Precision = \frac{TP}{TP + FP},
\end{equation}
where $TP + FP$ represents the total number of objects to detect. Otherwise, recall measures the goodness of detections performed by the model, relating the positive classifications with the total number of model's predictions.

\begin{equation}
\label{eq:recall}
Recall = \frac{TP}{TP + FN},
\end{equation}
where $TP + FN$ is the total number of predictions performed by the model. 
To discriminate true from false positive, detections (the predicted bounding boxes) are assigned to  ground-truth objects and judged to be true/false by measuring bounding box overlap. To be considered a correct detection, the area of overlap $a_0$ between the predicted bounding
box $B_p$ and  ground-truth bounding box $B_{gt}$ must exceed a threshold, set as 0.5 in the Pascal VOC challenge. This is the intersection over union (IoU) value:

\begin{equation}
\label{eq:iou}
a_0 = \frac{area(B_p \cap B_{gt})}{area(B_p \cup B_{gt})},
\end{equation}
where $B_p \cap B_{gt}$ denotes the intersection of the predicted and
 ground-truth bounding boxes and $B_p \cup B_{gt}$ their union. The area under the precision/recall curve is computed by is defined as the mean precision at a set of 11 equally spaced recall levels $ L = \{0, 0.1, 0.2, ..., 1\}$:

\begin{equation}
\label{eq:11-point}
	\text{AP} = \frac{1}{11} \sum_{r \in L} p_{interp}(r).
\end{equation}
The precision at each recall level $r$ is interpolated by taking
the maximum precision measured for which the corresponding recall exceeds $r$:

\begin{equation}
p_{interp} (r) = \max_{\tilde{r} : \tilde{r} \geq r} (\tilde{r}),
\end{equation}
where $p(\tilde{r})$ is the measured precision at recall $\tilde{r}$. 

The dataset of this thesis is composed of positive and negative images, that are treated differently by the evaluation method we propose. In the following paragraphs, we explain in detail how we evaluate the model's performance with the two different macro-categories of examples we collected (positive and negative). The evaluation procedure is implemented in a dedicated class, called \textsf{MyEvaluator}, contained in the main repository of this thesis. 

\paragraph{Positive Images} The positive images are examples that contain doors to detect. As described in Sec. \ref{sec:door_definition}, we consider two possible statuses that a door can assume: \textsf{open} and \textsf{closed}. This means that the bounding boxes in the positive images belong to a set of two different object categories: $L = \{0, 1\}$. To discriminate the background bounding boxes, the doors detector we propose does not output a third label, but it assign always a label in $L$ with  low accuracy, near to zero. 

The metric we use to evaluate the model's performance in the positive image is the same evaluation method used in Pascal VOC challenge \cite{pascal}. First of all, each image in the test set is classified by the model, and an instance of \textsf{MyEvaluator} stores the  ground-truth and the predicted bounding boxes. Then, the predicted bounding boxes are divided according to the object category the model assigns to them, in order to calculate an AP (average precision) score for each label (\textsf{open} and \textsf{closed} doors). The AP value is calculated as follow:

\begin{enumerate}
	\item the ground-truth and the predicted bounding boxes are filtered according to their confidence value: if a detection has confidence less than a threshold, called \textsf{confidence\_threshold}, it is discarded;
	\item the remaining bounding boxes are then descending ordered according to their confidence value;
	\item  following the order determined in the previous step, the module tries to match each predicted bounding box with a  ground-truth box. The matching procedure follows the next operations:
	\begin{itemize}
		\item each predicted bounding box is compared with the  ground-truth boxes of the same image, in order to find the one with the greater intersection under union (IoU) area (Eq. \ref{eq:iou});
		\item if the greater IoU area is overcome a threshold, called \textsf{iou\_threshold}, the predicted bounding box is matched with the corresponding  ground-truth box (if it has not been previously matched);
		\item the matching procedure determines the true positive (TP) and the false positive (FP) detections: each predicted bounding box matched with a  ground-truth box is considered a TP, while a detected bounding box with no match represents a FP;
	\end{itemize}
	\item when the matching procedure ends, the  ground-truth boxes not matched  are considered as false negative (FN);
	\item the true and false positives found during the matching operations are saved in order to compute the values of precision (Eq. \ref{eq:precision}) and recall (Eq. \ref{eq:recall}) at each matching step; 
	\item the final AP score is the area under the precision/recall curve, calculated by interpolating such curve at every point (refining the 11-point interpolation performed in Pascal VOC metric expressed in Eq. \ref{eq:11-point}).
\end{enumerate} 

\begin{figure}[h!]
	\centering
	\includegraphics[width=0.8\textwidth]{images/interpolated.png}
	\caption{The interpolated precision/recall curve. The blue curve is the original curve while the red one is the interpolation. AP is the area under the interpolated curve.}
	\label{fig:interpolation}
\end{figure}

\paragraph{Negative Images} The negative images do not contain objects of interest, so they are evaluated in a different way. The procedure is inspired by the metric of Pascal VOC challenge \cite{pascal} with some refinements. As mentioned in Sec. \ref{sec:detr}, DETR outputs a fixed number of predictions for every image. The back bounding boxes do not have a dedicated label, but they are characterized by low accuracy. The First of all, the predicted bounding boxes in the negative images lost their classic labels (\textsf{closed} or \textsf{open} doors). Then, the bounding boxes with a low confidence value are considered as TP (true positive) while those with a high confidence are FP (false positives). Finally, the metric we propose calculates the AP score on the average/precision curve. More in detail, the AP value in the negative images is computed as follow:

\begin{enumerate}
	\item the bounding boxes in the negative images lost the labels assigned by the model and assume a unique label that indicates all the negative boxes;
	\item the bounding boxes are grouped per image in which they are predicted;
	\item the negative images are ordered according to the sum of confidence of the predicted bounding boxes;
	\item following the order defined in the previous step, the evaluation method now processes each negative image to find the TP and the FP bounding boxes. If a bounding box has a confidence value less than \textsf{confidence\_threshold}, it is considered a true positive. Otherwise, a predicted bounding box in a negative image with a confidence score greater than \textsf{confidence\_threshold} is a false positive because it can be changed for a door;
	\item the true and false positives are used to compute the values of precision (Eq. \ref{eq:precision}) and recall (Eq. \ref{eq:recall}) at each step; 
	\item the final AP score is the area under the precision/recall curve, calculated by interpolating such curve at every point (refining the 11-point interpolation performed in Pascal VOC metric expressed in Eq. \ref{eq:11-point}).
\end{enumerate}






             % Concept Preview
\chapter{Experimental Results}
\label{sec:chapter5}
\thispagestyle{empty}

This chapter reports the experimental settings and the results obtained by this thesis. At first, we describe more in detail the data acquisition procedure, specifying the parameters settings of Pose Estimator (Sec. \ref{sec:pose_estimator}) and the final dataset composition. We proceed by describing the configuration of DETR we use for building the proposed doors detector: the architecture chosen from those proposed by the authors of \cite{detr}, the hyper-parameters values, and the weights' initialization. Then, we report a preliminary test of DETR with a well-known doors dataset: DeepDoors2 \cite{deepdoors2}. We discuss the results of DETR with this dataset to verify if it obtains acceptable performance in a doors detection task and if DETR converges with a training dataset smaller than COCO \cite{coco}. We also describe how we modify DeepDoors2  to fix some labeling errors. Finally, we evaluate the proposed doors detector with the doors dataset acquired in this thesis, exposing also an experimental evaluation of the \textbf{one-shot incremental learning} technique to measure the performance increasing of a general doors detector fine-tuned with new examples of a specific environment. 

\section{Dataset Acquisition}
\label{sec:dataset_acquisition}
The visual dataset has been collected in simulated environments from Matterport3D dataset \cite{matterport} through the modified version of Gibson \cite{gibson} (described in Sec. \ref{sec:new_gibson_version}). The entire dataset is publicly available\footnote{The dataset is downloadable \href{https://drive.google.com/file/d/1BqjBpobjKTomFjDkzhWjmCryAXOEluO2/view?usp=sharing}{here}.}. The positions from which the examples are acquired are extracted from the Pose Selector module, described in Sec \ref{sec:pose_estimator}. We select a set $E$ of $10$ worlds of Matterport3D dataset, where $E = \{\text{house1}, \text{house2}, \text{house7}, \text{house9}, \text{house10}, \text{house13}, \text{house15}, \text{house20},  \text{house21},  \text{house22}\}$. The dataset's acquisition procedure, previously described in Sec. \ref{sec:dataset_labeling}, is now formalized by reporting the various steps and specifying the hyper-parameters values. The collection algorithm works as follows:

\begin{enumerate}
	\item we start a simulation run with Gibson for each environment $e \in E$, where the virtualized agent is a Turtlebot2 \cite{turtlebot2};
	\item for each environment we select a set of positions through the Pose Estimator module, by setting \textsf{interval} $= 1.00m$. This means that the selected positions are spaced by $1.00m$ from each other in the Voronoi graph created by Pose Estimator;
	\item during the simulation run, the robot is positioned in each location and we collect a pool of examples changing the robot orientation and the height with respect to the floor. We select a set of height values $H = \{0.10m, 0.70m\}$ and a set of 8 rotation angles $O = \{0^{\circ}, 45^{\circ}, 90^{\circ}, 135^{\circ}, 180^{\circ}, 225^{\circ}, 270^{\circ}, 315^{\circ}\}$;
	\item for each position, we collect an example for every height-orientation pair in the Cartesian product $H \times O = \{(h, o) \mid h \in H, o \in O\}$. Since $|H \times O| = 16$, we collect 16 examples for each location extracted in the previous step;
	\item as mentioned in Sec \ref{sec:dataset_labeling}, an example is initially composed by the data provided by Gibson: an RGB image, depth data, and semantic annotations contained in another RGB image;
	\item as mentioned in Sec. \ref{sec:dataset_labeling}, the automatic labeling procedure exploiting the semantic images lead to errors that can degrade  the detector's performance. For fixing it, a human operator parses all the positive examples with a dedicated tool in order to fix the bounding boxes found with the semantically annotated images, to specify the label (which means \textit{close} or \textit{open} doors) for each bounding box, and to highlight the implicit doors which are not tagged. During this manual labeling procedure, the user also discards the bounding boxes around not valid objects of interest, that are doors too close or too far in the RGB image. In particular, we discarded a door too close to the acquisition position considering the depth data: if the average distance is less than $0.30 m$. Likewise, the doors too far with respect to the robot position are not considered if they cover less than $2,5\%$ of the total semantic image. If an image has no valid doors, the entire example is not considered. Another important goal of the manual labeling procedure is to remove the corrupted images produced by Gibson, as it contains several artifacts;
	\item with the same tool, a human operator checks also the negative images to discard the examples with wrong RGB images; 
	\item the labeling tool automatically saves the checked examples in a new dataset folder. Finally, each example is composed of an RGB image, the depth data, and an array that contains the bounding boxes coordinates and their labels (that indicate the door's status). 
\end{enumerate}

The dataset is managed using the \textit{generic-dataset} package described in Sec. \ref{sec:generic_dataset}. This framework organizes the dataset persistence directory-tree as described in Fig. \ref{fig:organization_dataset}.

\begin{figure}[h!]
	\centering
	\begin{minipage}{7cm}
		\dirtree{%
			.1 MAIN\_DATASET\_FOLDER.
			.2 house1.
			.3 1.
			.4 rgb\_image.
			.5 rgb\_image\_rc\_ac.png.
			.5 rgb\_image\_rc\_ac.png.
			.5 \vdots.
			.4 depth\_data.
			.5 \vdots.
			.4 bounding\_boxes.
			.5 \vdots.
			.3 0.
			.4 \vdots.
			.3 \vdots.
			.2 house2.
			.3 \vdots.
			.2 \vdots.
		}
	\end{minipage}
	\caption{The structure of the visual dataset collected in this thesis.}
	\label{fig:organization_dataset}
\end{figure} 

Table \ref{tab:dataset_examples_number} reports the number of examples acquired for each environment $e \in E$, that compose the dataset we collected.

\begin{table}[h!]
	\centering
	\begin{tabular}{cccc}

	\toprule
	\textbf{Env. name} & \textbf{Positive examples} & \textbf{Negative examples} & \textbf{Total
		examples}\tabularnewline
	\midrule
	house1 & 350 & 363 & 713\tabularnewline
	house2 & 482 & 529 & 1011\tabularnewline
	house7 & 358 & 227 & 585\tabularnewline
	house9 & 774 & 410 & 1184\tabularnewline
	house10 & 446 & 308 & 754\tabularnewline
	house13 & 413 & 230 & 643\tabularnewline
	house15 & 652 & 423 & 1075\tabularnewline
	house20 & 408 & 350 & 758\tabularnewline
	house21 & 826 & 551 & 1377\tabularnewline
	house22 & 748 & 515 & 1263\tabularnewline
	\bottomrule
	\end{tabular}
	\caption{The number of examples for every environment $e \in E$.}
	\label{tab:dataset_examples_number}
\end{table}
\newpage
\section{DETR Configuration}

As mentioned in Sec. \ref{sec:doors_detector}, the doors detector module proposed in this thesis is built using DETR \cite{detr}. As a reminder, DETR's architecture (described in Sec. \ref{sec:sec:detrarchitecture}) is composed of a CNN backbone (ResNet \cite{resnet}) which  provides a low dimensional representation of an image. Then, the features extracted are fed into a Transformer \cite{transformer} to capture the relationships between them by reasoning over the entire image as context. Finally, the bounding boxes' coordinates are inferred by a 4-layer perceptron while their labels are extracted through a linear classifier. In the following paragraphs, we describe in detail the configuration of DETR we use to run our experiments.

\subsection{DETR Architecture}
\citeauthor{detr} \cite{detr} propose 4 versions of DETR. The first one is composed by a ResNet-50 while the second has a ResNet-101 as feature extractors. The authors called these models DETR and DETR-R101 respectively. Then, the authors propose other two architectures starting from both DETR and DETR-R101. Following the work proposed in \cite{fullyconvolutional}, the authors of DETR increase the feature resolution by
adding a dilation to the last stage of the backbone and removing a stride from
the first convolution of this stage. The corresponding models are called respectively DETR-DC5 and DETR-DC5-R101. This modification
increases the resolution by a factor of two, thus improving performance for small
objects, increasing the cost by 16x in the self-attentions of the encoder,
leading to an overall of 2x in computational cost. A full comparison of
FLOPs (number of floating-point operations per second), FPS (frame per second), and the number of parameters of these models are given in Table \ref{tab:detr_models_flops}. The authors calculate the FLOPS for the first 100 images in the COCO 2017 validation set using tool \textsf{flop\_count\_operators} from Detectron2 \cite{detectron2}. We use the smaller version of DETR to build the doors detector: the model's efficiency is crucial in a robotic context and the modified versions of DETR increase a lot the inference time and the memory consumed. 

\begin{table}[h!]
	\centering
	\begin{tabular}{cccc}
		
		\toprule
		\textbf{Model name} & \textbf{GFLOPS} & \textbf{FPS} & \textbf{Paramaters} \tabularnewline
		\midrule
		DETR & 86 & 28 & 41M\tabularnewline
		DETR-DC5 & 187 & 12 & 41M\tabularnewline
		DETR-R101 & 152 & 20 & 60M\tabularnewline
		DETR-DC5-R101 & 233 & 10 & 60M\tabularnewline
		\bottomrule
	\end{tabular}
	\caption{The comparison between the four architecture of DETR. Table from \cite{detr}.}
	\label{tab:detr_models_flops}
\end{table}

\subsection{DETR's Configuration}
\label{sec:detr_configuration}
We do not retrain the entire model but we load the pre-trained version of DETR furnished by the authors. This is because training DETR from scratch is unfeasible. First of all, as reported in \cite{surveytransformer}, the Transformers used in Computer Vision need wide training datasets. As a prove, DETR is trained for 300 epochs using the COCO 2017 dataset \cite{coco}, which contains about 118K of training images. This procedure takes 3 days on a cluster with 16 Tesla V100 GPUs and a batch size of 64 (4 images for each GPU). Since we have a small doors dataset (with about 8K examples) and a limited computing power, we fine-tune DETR with a few data to resolve a more refined task (detecting doors). Fine-tuning a model pre-trained with a dataset like Imagenet \cite{imagenet} has become a common technique for solving computer vision tasks \cite{verydeepimagenet, resnet, fasterrcnn, yolo, yolov2}.

\subsection{DETR Hyper-Parameters}
Now, we describe in detail the setting of the hyper-parameters used for training the model. We train DETR using the AdamW  \cite{adamw} optimizer implemented in PyTorch. We set AdamW with a \textsf{weight\_decay} of $1^{-4}$. The backbone and the transformers are treated slightly differently.  We train the CNN backbone with a learning rate of $10^{-6}$, while the learning rate of the Transformer is set at $10^{-5}$. The authors of \cite{detr} observe that having the backbone learning rate roughly an order of magnitude smaller than the rest of the network is important to stabilize training, especially in the first few epochs. 

As reported in Sec. \ref{sec:detrlosses}, the loss function for bounding box regression is a linear combination of $\ell_1$ and generalized IoU \cite{generalizediou} losses (Eq. \ref{eq:bounding_box_loss}). We set $\lambda_{iou} = 2$ and $\lambda_{L1} = 5$.

Another important hyper-parameter is the number of object queries. As specified in Sec. \ref{sec:detrarchitecture}, the module produces a detection for each object query. In the original article of DETR \cite{detr}, the total number of object queries is  $N = 100$. This is because, as specified by the authors, $N$ must be greater than the maximum number of objects instances in an image (the images of COCO contain up to 70 distinct objects). In our dataset, the maximum number of doors in an image is 3, so we set $N = 10$.

\subsection{Data Augmentation} DETR \cite{detr} is a wide model that requires a huge amount of training data to converge. To overcome this issue, the authors perform an intense data augmentation of the COCO's images used for DETR's training. Another purpose of the data augmentation technique is to generalize well the problem by producing new images starting from the original ones. In this way, the model learns from different images increasing in accuracy and preventing overfitting. 

The data augmentation applied to each training image is composed by the following operations:

\begin{enumerate}
	\item \textbf{horizontal flip:} at first, the authors apply a horizontal flip to the image with a probability of $0.5$;
	\item \textbf{random select:} now, the data augmentation proceeds choosing one of the following algorithms with a probability of $0.5$:
	\begin{enumerate}
		\item \textbf{random resize:} the image is randomly resized such that the shortest side is at least 480 and at most 800 pixels while the longest at most 1333;
		\item \textbf{random size crop:} the image, which is initially resized with the same procedure of the previous operation, is cropped to a random rectangular patch (with random sizes) which is then resized again;
	\end{enumerate} 
	\item \textbf{normalization:} normalize an image means transforming it into such values that the mean and standard deviation become 0.0 and 1.0 respectively. First of all, an image $W \times H \times 3$ is converted to a tensor with shape $3 \times H \times W $ and the integer pixels are scaled in the $[0.0, 1.0]$ interval. Then, each input channel is subtracted by the channel mean and then the result is divided by the channel standard deviation. The channels' mean and standard deviation used by the authors are calculated over the Imagenet dataset \cite{imagenet}.
\end{enumerate}

During the first experiments, we argue that this massive data augmentation is not appropriate in the context of this thesis. First of all, we use the pre-trained DETR version on COCO 2017 dataset, so the model already has a good initialization of the weights. Furthermore, we use a dataset less than one order of magnitude with respect to COCO. This means that we do not have enough examples for making a data modification like those performed in the original train of DETR. Another important issue regards the cropping procedure with respect to our objects of interest. In the context of this thesis, we aim to detect open or closed doors, that are typically big objects in images. By randomly cropping a frame, the door's features may be insufficient for a successful model's training. For these reasons, we perform a reduced data augmentation. With a probability of $0.5$, we modify the image with a random horizontal flip followed by a random resize operation; otherwise, the image is not modified. Finally, we normalize the frame with the mean and standard deviation of the Imagenet dataset. Since our images are smaller ($256 \times 256 \times 3$) than those of COCO, the random resize is performed in a different range, between 256 and 576 pixels. 

\section{DETR Analysis}

Before proceeding with the evaluation of DETR over the collected dataset, we test the proposed doors detector with a well-known doors dataset, called \textit{DeepDoors2} \cite{deepdoors2}, which is freely available on Github\footnote{The DeepDoors2 Github page: \url{https://github.com/gasparramoa/DeepDoors2}.}. We perform this experiment to understand if DETR is trainable with a smaller dataset than COCO and to verify if it obtains good results in a doors detection task. In the following sub-sections, we describe in detail the DeepDoors2 dataset proposed in \cite{deepdoors2} and how we modify it to better approach the requirements of the task addressed by this thesis. Then, we report the results obtained by the doors detector we propose if trained with the improved version of DeepDoors2. Finally, we further analyze the DETR's performance by visualizing the detections produced by the trained model using t-SNE \cite{tsne}.

\section{DeepDoors2 Dataset}
DeepDoors2 \cite{deepdoors2} is a visual dataset that contains 3000 labeled examples composed of  RGB images (with a dimension of $480 \times 640 \times 3$), depth data, and the relative semantic information encoded in others RGB images. The labels provided by the authors indicates also the doors' status, which can be open, semi-open, and closed (Fig. \ref{fig: open_semi_closed}). The examples are equally divided over these doors' categories.  This dataset is constituted of 3 parts, a 2D and 3D image classification part, a semantic segmentation part, and an object detection part. For the first two parts, the authors use their previous
dataset, published in \cite{deepdoors1}, and improve it by collecting more data and annotating more images. The third part was built by labeling the image of the classification part. This dataset was captured in different indoor environments (universities, public spaces, and houses) using a portable system constituted of a Raspberry Pi 3 B+ with a 3D Realsense Camera, model D435. The authors captured several images of doors and their surroundings with different textures and sizes, sometimes obstructed by obstacles (e.g. chairs, tables, furniture, and even persons). The authors also changed the pose to get different perspectives on the same door. 

\begin{figure}[h!]
	\hfil
	\begin{subfigure}[b]{0.21\linewidth}
		\includegraphics[width=\linewidth]{images/deep_doors_2_open.png}
		\caption{}
	\end{subfigure}
	\hfil
	\begin{subfigure}[b]{0.21\linewidth}
		\includegraphics[width=\linewidth]{images/deep_doors_2_semiopen.png}
		\caption{}
	\end{subfigure}
	\hfil
	\begin{subfigure}[b]{0.21\linewidth}
		\includegraphics[width=\linewidth]{images/deep_doors_2_closed.png}
		\caption{}
	\end{subfigure}
	\hfil
	\caption{Labeled images from DeepDoors2 dataset. (a) An open door, (b) a semi-open door, and (c) a closed door. Green, blue, and red bounding boxes represent open, semi-open, and closed doors respectively.}
	\label{fig: open_semi_closed}
\end{figure}

The object detection part of DeepDoors2 dataset is annotated with an automatic procedure that finds bounding boxes around doors using the semantic data. The labeling algorithm implemented by the authors finds a single door for each image, but we argue that a single example can depict multiple doors instances with different statues (Fig. \ref{fig:relabeling_deepdoors2}). Due to this fact, we manually re-label the entire dataset with the visual tool described in Sec. \ref{sec:dataset_labeling}. Furthermore, we re-organize the dataset according to the standard defined by the \textit{generic-dataset} framework reported in Sec. \ref{sec:generic_dataset}. We release the re-labeled version of DeepDoors2 dataset\footnote{The DeepDoors2 re-labeled dataset can be downloaded \href{https://drive.google.com/file/d/1wSmFUHF9aSJkomwFdOmepMevBvkRpf3D/view?usp=sharing}{here}.} and the necessary Python code\footnote{The DeepDoors2 re-labeled source code: \url{https://github.com/micheleantonazzi/deep-doors-2-labelled}.} to manage it. The final DeepDoors2 dataset is composed of 2998 examples, while each of them includes an RGB image with dimension $480 \times 640 \times 3$, a matrix $480 \times 640$ which contains the depth data, and a list with the bounding boxes' coordinates and the relative labels (open, semi-open, or closed door). 

\begin{figure}[h!]
	\centering
	\begin{subfigure}[b]{\linewidth}
		\hfil
		\begin{subfigure}[b]{0.21\linewidth}
			\includegraphics[width=\linewidth]{images/deep_doors_2_labeling1.png}
		\end{subfigure}
		\hfil
		\begin{subfigure}[b]{0.21\linewidth}
			\includegraphics[width=\linewidth]{images/deep_doors_2_labeling1_semantic.png}
		\end{subfigure}
		\hfil
		\begin{subfigure}[b]{0.21\linewidth}
			\includegraphics[width=\linewidth]{images/deep_doors_2_labeling1_correct.png}
		\end{subfigure}
		\hfil
		\caption{}
	\end{subfigure}
	\newline
	\begin{subfigure}[b]{\linewidth}
		\hfil
		\begin{subfigure}[b]{0.21\linewidth}
			\includegraphics[width=\linewidth]{images/deep_doors_2_labeling2.png}
		\end{subfigure}
		\hfil
		\begin{subfigure}[b]{0.22\linewidth}
			\includegraphics[width=\linewidth]{images/deep_doors_2_labeling2_semantic.png}
		\end{subfigure}
		\hfil
		\begin{subfigure}[b]{0.21\linewidth}
			\includegraphics[width=\linewidth]{images/deep_doors_2_labeling2_correct.png}
		\end{subfigure}
	\hfil
		\caption{}
	\end{subfigure}
	\caption{Two examples of wrong labeled images from the original version of DeepDoors2 dataset. The left figures in each row represents the bounding boxes found with the depth data encoded in the middle images. The figures on the right depict the fixed labeled images of our version DeepDoors2.}
	\label{fig:relabeling_deepdoors2}
\end{figure}

\subsection{DETR's Performance on DeepDoors2}

Before proceeding with other experiments, we test DETR on the relabeled version of DeepDoors2 to verify if it converges with such a small dataset and if it has acceptable performance in a doors detection task. We randomly split the dataset into a training set with the $80\%$ of the examples and a test set with the remaining. We train DETR for 40 epochs with a batch size of 1. We report the DETR loss function (Eq. \ref{eq:detr_loss}) during training both for train and test sets in Fig. \ref{fig:deep_doors2_loss}.

\begin{figure}[h!]
	\centering
	\includegraphics[width=\linewidth]{images/deep_doors_2_loss.png}
	\caption{The loss functions during the training of DETR.}
	\label{fig:deep_doors2_loss}
\end{figure}

We evaluate the trained version of DETR with the Pascal VOC metric \cite{pascal} as described in Sec. \ref{sec:model_evaluator}. We set the \textsf{iou\_threshold}$ = 0.9$ and the \textsf{confidence\_threshold} $= 0.5$. Table \ref{tab:deep_doors2_results} reports the AP score for each object category (open, semi-open, and closed doors) and the data to calculate it, such as the total number of the ground-truth bounding boxes, the true positive, and the false positive detections performed by the model. The plot of the interpolated precision/recall curves for each category is shown in Fig. \ref{fig:deep_doors2_ap_plot}.

\begin{table}[h!]
	\centering
	\begin{tabular}{cccccc}
		
		\toprule
		\textbf{Label} & \textbf{AP} & \textbf{N. Positives} & \textbf{TP} & \textbf{FP} & \textbf{FN}\tabularnewline
		\midrule
		Closed door (0) & 90 & 234 & 214 & 45 & 20 \tabularnewline
		Semi-open door (1) & 83 & 198 & 169 & 33 & 29 \tabularnewline
		Open door (2) & 85 & 243 & 214 & 66 & 29 \tabularnewline
		\bottomrule
	\end{tabular}
	\caption{The performance of DETR trained on the DeepDoors2 dataset.}
	\label{tab:deep_doors2_results}
\end{table}

\begin{figure}[h!]
	\centering
	\includegraphics[width=0.93\linewidth]{images/deep_doors_2_precision_recall.png}
	\caption{The interpolated precision/recall curves about open, semi-open, and closes doors, colored in green, blue, and red respectively.}
	\label{fig:deep_doors2_ap_plot}
\end{figure} 

As shown by Fig. \ref{fig:deep_doors2_loss}, the model converges correctly and does not overfit or underfit. Both the training and test error (reported in blue and orange respectively) are low and closed to each other. As reported in Table \ref{tab:deep_doors2_results}, the proposed doors detector reaches a good detection accuracy for all the 3 door categories (open, semi-open, and closed). An AP greater than 80 is considered a good result in the Computer Vision community.

\subsection{The DETR's Detection Visualization}
As mentioned in Sec. \ref{sec:detrarchitecture}, each object query produces a detection composed of the bounding box coordinates and the relative label. To further analyze the DETR's performance, we plot in 2D the object queries classified by the Transformers using t-SNE \cite{tsne}. t-SNE (t-distributed Stochastic Neighbor Embedding) is a unsupervised and randomized technique to visualize high-dimensional data by giving each data point a location in a two or three dimensional map. This algorithm is a variation of Stochastic Neighbor Embedding (SNE) \cite{sne}. We plot the output embeddings of the Transformer with t-SNE in a two-dimensional space. We train DETR for 40 epochs with a batch size of 1 using 80\% of the examples of the re-labeled version of DeepDoors2. Then, we classify both the training set and the test set (composed the 20\% of the remaining examples) with the trained model, saving the object queries modified by the Transformer. Finally, we cluster only the predictions with the highest accuracy of every image using the \textit{scikit-image} implementation of t-SNE, by setting 2 different values for perplexity: 30 and 100. This hyper-parameter is a guess about the number of close neighbors each point has. Before clustering them through t-SNE, we reduce their dimensionality to 50 using the Principal Components Analysis (PCA) \cite{pca} algorithm, which projects a data point onto only the first few principal components preserving as much of the data's variation as possible. 

\begin{figure}[h!]
	\centering
	\includegraphics[width=\linewidth]{images/deep_doors_2_tsne_trainset.png}
	\caption{The plot of the Transformer's output embeddings using t-SNE over the training set.}
	\label{fig:tsne_deep_doors2_train}
\end{figure}

\begin{figure}[h!]
	\centering
	\includegraphics[width=\linewidth]{images/deep_doors_2_tsne_testset.png}
	\caption{The plot of the Transformer's output embeddings using t-SNE over the test set.}
	\label{fig:tsne_deep_doors2_test}
\end{figure}
 
The Figs. \ref{fig:tsne_deep_doors2_train} and \ref{fig:tsne_deep_doors2_test} show that DETR produces useful features encoding to separate the open, semi-open, and closed doors. Thanks to t-SNE, we argue that the objects queries classified by the Transformer are clearly separated into 3 distinct clusters examining both the training and the test sets (as expected looking at the good results reported in Tab. \ref{tab:deep_doors2_results}). Since the AP scores are not equal to 100, there are some spurious detections in the t-SNE's plots that likely lead to false positives.

\section{One-Shot Incremental Learning Evaluation}

As mentioned in Sec. \ref{sec:solution}, we not only offer a deep learning module for finding doors, but we also propose a technique for increasing the detector's performance exploiting a specific deployment scenario for a mobile robot as  explained in Sec. \ref{sec:deploymentscenario}. This technique, which we call \textbf{one-shot incremental learning}, consists on fine-tune a generic doors detector with a new dataset acquired directly from the new environment.
To successfully evaluate the \textbf{one-shot incremental learning} approach, the experimental phase of this thesis emulates the requirements for ideal deployment scenario for a mobile robot using the dataset of doors we collected. As reported in Sec. \ref{sec:dataset_acquisition}, the visual doors dataset has been acquired from set $E$ of 10 different environments from Matterport3D \cite{matterport} virtualized through the modified version of Gibson \cite{gibson} (described in Sec. \ref{sec:new_gibson_version}). The examples are both positive and negative and they are divided according to the environment of belonging. For each environment $e \in E$, the dataset $D_{e}$ (which contains the examples acquired from $e$) is split into 2 sub-sets $G_e$ and $S_e$, where $G_e$ contains all the examples that do not belong to the selected environment $e$, while $S_e$ is composed of the images collected from $e$. Then, $G_e$ is divided into 2 sub-sets $G^{P}_e$ and $G^{N}_e$: the first contains only the positive examples while the latter only the negative ones not acquired in the environment $e$. Likewise, $S_e$ is split in $S^{P}_e$ and $S^{N}_e$ that respectively contain the positive and the negative images of the environment $e$. Finally, 4 sub-sets are extracted from $S^{P}_e$, which are $S^{P}_{e, 1}$, $S^{P}_{e, 2}$, $S^{P}_{e, 3}$, and $S^{P}_{e, 4} $. They are composed of the 25\% of the examples of $S^{P}_e$. We use this dataset division to perform a series of two experiments for each environment:

\begin{itemize}
	\item \textbf{Experiment 1:} at first, we train for 40 epochs a general doors detector with $G^{P}_e$, that contains the positive examples that do not belong to the environment $e$. We use the configuration of DETR reported in Sec. \ref{sec:detr_configuration}. Then, we evaluate this model using the metric explained in Sec. \ref{sec:model_evaluator}, with a test set composed by $S^{P}_{e, 4}$ and $S^{N}_e$, that respectively contain the 25\% of the positive and all the negative image collected in $e$. We formally call this experiment \textsf{GD} (general doors detector). 
	
	\item \textbf{Experiment 2:} then, we perform a series of 3 fine-tune operations on the module trained in the previous experiment. We re-train (using the parameters reported in Sec. \ref{sec:detr_configuration}) for 20 epochs  the pre-trained module producing 3 fine-tuned versions of the generic doors detector using three different training sets $T$. At first, we set $ T= \big\{S^{P}_{e, 1}\big\}$, then $T=\big\{S^{P}_{e, 1} \cup S^{P}_{e, 2}\big\}$, and finally $T=\big\{S^{P}_{e, 1} \cup S^{P}_{e, 2} \cup S^{P}_{e, 3}\big\}$, so we fine-tune the general model by employing the  25\%, 50\%, and the 75\% of the positive samples collected in $e$. Like the previous experiment, each of the fine-tuned doors detector is evaluated through the metric described in Sec. \ref{sec:model_evaluator} using the remaining 25\% of the positive examples and all the negative images, contained in $S^{P}_{e, 4}$ and $S^{N}_e$ respectively. We refer to these 3 fine-tuned detectors as \textsf{FD\textsubscript{25}}, \textsf{FD\textsubscript{50}}, and \textsf{FD\textsubscript{75}}. 

\end{itemize}

In simple terms, we built 4 different doors detector for each environment in the collected dataset: the general doors detector (\textsf{GD}) is trained without images acquired in the selected environment, and other 3 detectors trained by fine-tuning the general model with the 25\% (\textsf{FD\textsubscript{25}}), 50\% (\textsf{FD\textsubscript{50}}), and 75\% (\textsf{FD\textsubscript{75}}) of the examples of the selected environment. All these models are tested with the remaining 25\% of the samples of the current environment ($S^{P}_{e,4}$) and all the negative images of the current environment ($S^{N}_{e}$).

\subsection{Detail Results in One Example Environment}

Before proceeding with a complete analysis of the proposed method using all the environments in $E$, we report a detailed study of a single environment from Matterport3D \cite{matterport}, \textsf{house13}. First of all, we test the performance of the trained detectors (the general detector \textsf{GD} and the 3 fine-tuned models \textsf{FD\textsubscript{25}}, \textsf{FD\textsubscript{50}}, and \textsf{FD\textsubscript{75}}) using the metric explained in Sec. \ref{sec:model_evaluator}. Table \ref{tab:house13_results} shows the results obtained with \textsf{house13}, reporting the AP scores and the increment in percentage of AP between each experiment and the previous one. The AP values are computer on a test set $T = S^{P}_{e, 4} \cup S^{N}_{e}$, which contains the 25\% of positive and all the negative examples collected in $e$. The values contained in this table are plotted in Fig. \ref{fig:house13_AP_results}, which reports the AP scores grouped by label in order to visualize the AP change from one experiment to the next. We set the metric's hyper-parameters at 0.5 and 0.75 for the \textsf{confidence\_threshold} and \textsf{iou\_threshold} respectively.

\begin{table}[h!]
	\centering
	\begin{tabular}{ccccc}
		
		\toprule
		\textbf{Env.} & \textbf{Exp.} & \textbf{Label} & \textbf{AP} & \textbf{Increment}  \\
		\midrule
		\multicolumn{1}{c|}{\multirow{12}{*}{House 13}} & \multicolumn{1}{c|}{\multirow{3}{*}{\textsf{GD}}} & No door (-1) & 68 & -  \tabularnewline 
		\multicolumn{1}{c|}{}& \multicolumn{1}{c|}{} & Closed door (0) & 66 & -  \\
		\multicolumn{1}{c|}{}& \multicolumn{1}{c|}{}& Open door (1) & 64 & -  \\  \cline{2-5}
		\multicolumn{1}{c|}{}& \multicolumn{1}{c|}{\multirow{3}{*}{\textsf{FD\textsubscript{25}}}} & No door (-1) & 74 & $+8\%$  \tabularnewline [1pt]
		\multicolumn{1}{c|}{}& \multicolumn{1}{c|}{} & Closed door (0) & 74 & $+13\%$  \\ 
		\multicolumn{1}{c|}{}& \multicolumn{1}{c|}{} & Open door (1) & 67 & $+4\%$   \\ \cline{2-5}
		\multicolumn{1}{c|}{} & \multicolumn{1}{c|}{\multirow{3}{*}{\textsf{FD\textsubscript{50}}}} & No door (-1) & 76 & $+2\%$  \tabularnewline 
		\multicolumn{1}{c|}{}& \multicolumn{1}{c|}{} & Closed door (0) & 84 & $+14\%$  \\
		\multicolumn{1}{c|}{}& \multicolumn{1}{c|}{}& Open door (1) & 76 & $+13\%$  \\  \cline{2-5}
		\multicolumn{1}{c|}{}& \multicolumn{1}{c|}{\multirow{3}{*}{\textsf{FD\textsubscript{75}}}} & No door (-1) & 79 & $+4\%$  \tabularnewline [1pt]
		\multicolumn{1}{c|}{}& \multicolumn{1}{c|}{} & Closed door (0) & 83 & $-2\%$  \\ 
		\multicolumn{1}{c|}{}& \multicolumn{1}{c|}{} & Open door (1) & 73 & $-4\%$   \\ 
		\bottomrule
	\end{tabular}
	\caption{The evaluation of the \textbf{one-shot incremental learning} on \textsf{house13}. This table reports the AP scores obtained by the 4 doors detectors on 3 labels: negative bounding box (predicted in the negative images), closed doors, and open doors. In addition, the last column shows the increments obtained by the 3 fine-tune operations.}
	\label{tab:house13_results}
\end{table}

\begin{figure}[h!]
	\centering
	\includegraphics[width=0.86\linewidth]{images/house13_AP_results.png}
	\caption{The AP score obtained in the experiments of \textsf{house13} grouped by label (negative bounding box, closed door, and open door).}
	\label{fig:house13_AP_results}
\end{figure}

Then, we report the plots of the loss functions (Eq. \ref{eq:detr_loss}) during the training of the 4 doors detectors (Fig. \ref{fig:house13_losses}). The training loss is the average of the loss value computed with the training data during each epoch, while the test loss is calculated at the end of each epoch using all the positive examples ($S^{P}_e$) in the general detector and the 25\% of positive images (which corresponds to the $S^{P}_{e,4}$ sub-set) for the 3 fine-tuned models.

\begin{figure}[h!]
	\centering
	\begin{subfigure}[b]{0.49\linewidth}
		\includegraphics[width=\linewidth]{images/house13_general_detector_loss.png}

	\end{subfigure}
	\hfil
	\begin{subfigure}[b]{0.49\linewidth}
		\includegraphics[width=\linewidth]{images/house13_finetune25_loss.png}

	\end{subfigure}
	\newline
	\newline
	\begin{subfigure}[b]{0.49\linewidth}
		\includegraphics[width=\linewidth]{images/house13_finetune50_loss.png}

	\end{subfigure}
	\hfil
	\begin{subfigure}[b]{0.49\linewidth}
		\includegraphics[width=\linewidth]{images/house13_finetune75_loss.png}

	\end{subfigure}
	\caption{The losses values of DETR during the training on \textsf{house13}. }
	\label{fig:house13_losses}
\end{figure}

The first consideration is that the configuration of DETR (reported in Sec. \ref{sec:detr_configuration}) allows  a correct training of all the model versions (\textsf{GD}, \textsf{FD\textsubscript{25}}, \textsf{FD\textsubscript{50}}, and \textsf{FD\textsubscript{75}}). As shown in Fig. \ref{fig:house13_losses}, both the training and test losses decrease to 0 and are close to each other. 

By examining the table reporting the AP score (Tab \ref{tab:deep_doors2_results}), we argue that our door detector achieves good performance even with a general train (without using any examples of {house13}). In fact, the \textsf{GD} detector reaches 68, 66, and 64 of AP for negative bounding boxes, closed doors, and open doors respectively. The fine-tune operations tend to increase the doors detector's performance. In particular, the first fine-tuned detector (\textsf{FD\textsubscript{25}}) increases the AP by a factor of 8\%, 13\%, and 4\% for the negative bounding boxes, closed doors, and open doors respectively. Also the \textsf{FD\textsubscript{50}} module increase a lot the doors detector's performance. More precisely, by a factor of 2\%, 14\%, and 13\% for negative bounding boxes, closed doors, and open doors respectively. Surprisingly, the detector fine-tuned with the 75\% of the positive examples of \textsf{house13} degrades the AP scores for both open and closed doors. As shown in Fig. \ref{fig:house13_AP_results}, the AP score of the negative bounding boxes increases in all experiments, but, for the closed and open doors, the AP value significantly grows up only in the \textsf{FD\textsubscript{25}} and \textsf{FD\textsubscript{50}}, while it goes down in the last fine-tuned detector (\textsf{FD\textsubscript{75}}).

\subsection{Results Over All Environments}

After the analysis of the \textbf{one-shot incremental learning} behavior considering a single environment (\textsf{house13}), we evaluate such a technique on the entire visual dataset collected in this thesis. As a reminder, the dataset has been acquired from a set $E$ composed of 10 environments of Matterport3D \cite{matterport}: \textsf{house1}, \textsf{house2}, \textsf{house7}, \textsf{house9}, \textsf{house10}, \textsf{house13}, \textsf{house15}, \textsf{house20}, \textsf{house21},  \textsf{house22}.
For each environment $e \in E$, we train 4 door detectors. The first (\textsf{GD}) is trained with all the examples of the dataset except those collected in the selected environment ($G^{P}_e$). Then, we fine-tune 3 times the general model with the 25\%, 50\%, and 75\% of the positive examples from $e$,  producing 3 fine-tuned detectors: \textsf{FD\textsubscript{25}}, \textsf{FD\textsubscript{50}}, and \textsf{FD\textsubscript{75}}. We perform their  evaluation using the metric describe in Sec. \ref{model_evaluator}, setting \textsf{iou\_threshold} $= 0.75$ and the \textsf{confidence\_threshold} $= 0.50$. The AP values are calculated using a fixed test set $T = S^{P}_{e, 4} \cup S^{N}_{e}$, composed by the 25\% of positive and all the negative examples collected in $e$.  We report the mean of the AP scores reached by the 4 detectors divided by label (negative bounding boxes, closed door, and open door), the increments' average obtained with the fine-tune operations, and the standard deviation ($\sigma$) of both of them in Table \ref{tab:all_houses_results}. We also report the AP scores of all environments in two plots. In particular, Fig. \ref{fig:all_houses_closed} depicts the AP scores reached by the general and the 3 fine-tuned detectors on closed doors, while in Fig. \ref{fig:all_houses_open} are reported the detectors' performance on open doors. The environments are increasing ordered based on the AP score obtained with the \textsf{FD\textsubscript{25}} doors detector.

\begin{table}[h!]
	\centering
	\begin{tabular}{ccccccc}
		
		\toprule
		\textbf{Env.} & \textbf{Exp.} & \textbf{Label} & \textbf{AP} & \textbf{$\sigma$} & \textbf{Increment} &  \textbf{$\sigma$} \\
		\midrule
		\multicolumn{1}{c|}{\multirow{12}{*}{All houses}} & \multicolumn{1}{c|}{\multirow{3}{*}{\textsf{GD}}} & No door (-1) & 74 & 5 & - & - \tabularnewline 
		\multicolumn{1}{c|}{}& \multicolumn{1}{c|}{} & Closed door (0) & 39 & 14 & - & -  \\
		\multicolumn{1}{c|}{}& \multicolumn{1}{c|}{}& Open door (1) & 60 & 8 & - & -  \\  \cline{2-7}
		\multicolumn{1}{c|}{}& \multicolumn{1}{c|}{\multirow{3}{*}{\textsf{FD\textsubscript{25}}}} & No door (-1) & 79 & 4 & $+7\%$ & $9\%$ \tabularnewline [1pt]
		\multicolumn{1}{c|}{}& \multicolumn{1}{c|}{} & Closed door (0) & 61 & 10 &  $+75.5\%$ & $66\%$  \\ 
		\multicolumn{1}{c|}{}& \multicolumn{1}{c|}{} & Open door (1) & 70 & 7 & $+16\%$ & $14\%$  \\ \cline{2-7}
		\multicolumn{1}{c|}{} & \multicolumn{1}{c|}{\multirow{3}{*}{\textsf{FD\textsubscript{50}}}} & No door (-1) & 78 & 5 & $-1\%$ & $5\%$  \tabularnewline 
		\multicolumn{1}{c|}{}& \multicolumn{1}{c|}{} & Closed door (0) & 69 & 12 & $+13\%$ & $9\%$ \\
		\multicolumn{1}{c|}{}& \multicolumn{1}{c|}{}& Open door (1) & 73 & 6 & $+5\%$ & $5\%$ \\  \cline{2-7}
		\multicolumn{1}{c|}{}& \multicolumn{1}{c|}{\multirow{3}{*}{\textsf{FD\textsubscript{75}}}} & No door (-1) & 79 & 5 & $+1\%$ & $4\%$  \tabularnewline [1pt]
		\multicolumn{1}{c|}{}& \multicolumn{1}{c|}{} & Closed door (0) & 72 & 11 & $+6\%$ & $10\%$  \\ 
		\multicolumn{1}{c|}{}& \multicolumn{1}{c|}{} & Open door (1) & 76 & 6 & $+5\%$ & $5\%$  \\ 
		\bottomrule
	\end{tabular}
	\caption{The evaluation of the \textbf{one-shot incremental learning} over the entire dataset. This table reports the mean of the AP scores of the 4 doors detectors on the 3 labels: negative bounding box (predicted in the negative images), closed doors, and open doors, showing also the  the increments' average obtained by the 3 fine-tune operations. In addition, we report the standard deviations for the AP scores and the relative increments.}
	\label{tab:all_houses_results}
\end{table}

\begin{figure}[h!]
	\centering
		\includegraphics[width=\linewidth]{images/all_houses_results_closed.png}
	\caption{The AP scores obtained by the 4 doors detectors' versions (\textsf{GD}, \textsf{FD\textsubscript{25}}, \textsf{FD\textsubscript{50}}, and \textsf{FD\textsubscript{75}}) over the entire dataset on closed doors, grouped by environment. }
	\label{fig:all_houses_closed}
\end{figure}

\begin{figure}[h!]
	\centering
	\includegraphics[width=\linewidth]{images/all_houses_results_open.png}
	\caption{The AP scores obtained by the 4 doors detectors' versions (\textsf{GD}, \textsf{FD\textsubscript{25}}, \textsf{FD\textsubscript{50}}, and \textsf{FD\textsubscript{75}}) over the entire dataset on open doors, grouped by environment.}
	\label{fig:all_houses_open}
\end{figure}

The results reported in Table \ref{tab:all_houses_results} shows that our doors detector obtains, in average, good AP scores for all the considered labels even if it is trained in a general way (without considering the examples of the selected environment). The \textsf{GD} detector recognizes negative bounding boxes, closed doors, and open doors with average AP scores of 74, 39, and 60 respectively. By observing the increment values, we argue that the fine-tuned versions of the general doors detectors increase in average the detection accuracy on all classes with respect to the previous steps. The unique exception is for the negative bounding boxes in \textsf{FD\textsubscript{50}} detectors, where the average AP score decreases by 1\%.  The table shows that the first fine-tune operation (in which the general detectors are trained with the 25\% of the examples collected from a certain environment) gives the highest increment in performance. The closed doors, which is the category worst recognized by the general detectors, reaches a mean AP of 61, with a growth of 75\%. The AP averages of the open doors and the negative bounding boxes reach the values of 70 and 79 respectively, with an increment of 16\% and 7\%. The subsequent fine-tune operations (\textsf{FD\textsubscript{50}} and \textsf{FD\textsubscript{75}}) increase the average detection accuracy on all labels (except for negative bounding boxes in \textsf{FD\textsubscript{50}}) but in much smaller increments with respect to the fine-tune with only the 25\% of examples. 

Figs. \ref{fig:all_houses_closed} and \ref{fig:all_houses_open} allow us to better evaluate the impact of the \textbf{one-shot incremental learning} approach applied to all the environments in the collected dataset. They clearly show that the performance of the general detectors over the closed doors is extremely variable (as shown by the standard deviations in Table \ref{tab:all_houses_results}) and, sometimes, the closed doors are detected with an AP less than 30 (for example in \textsf{house2}, \textsf{house10}, and \textsf{house20}). At best of our knowledge, this fact can be caused by two factors. The first is the dataset's imbalance: the closed doors' examples are significantly less than open doors' images. For example, \textsf{house2} has 155 images of open doors but only 30 of closed doors. The second reason is the features that characterize the closed doors. Such a door category is stronger to be detected because it often blends with the surrounding wall. Otherwise, the doors detector works well with open doors because they expose more clear features. These figures confirm that fine-tuning a general doors detector with the 25\% of examples collected in a specific environment assure the best performance increasing with respect to the subsequent fine-tunes for both open and closed doors. Despite the \textsf{FD\textsubscript{75}} detectors reach in average the best performance, for many environments this is not true. For example, considering the open doors, in \textsf{house15} and \textsf{house1} fine-tuning with the 75\% of examples does not increase the module's performance, while in \textsf{house13} \textsf{FD\textsubscript{75}} degrades the detection accuracy with respect to \textsf{FD\textsubscript{50}}. Likewise, fine-tuning with the 50\% of the positive examples generally increase the performance with respect to the previous steps, but this fact is not confirmed by \textsf{house7} and \textsf{house10} considering the open doors.

The results reported in this section show that the collected dataset is more challenging than DeepDoors2 \cite{deepdoors2} and it is more suitable for evaluating a detector of doors used by a mobile robot. This is because it allows reproducing the deployment scenario described in Sec. \ref{sec:deploymentscenario}. The examples are divided according to the environment from which they belong, so it allows to evaluate the detection accuracy of an end-to-end module for finding doors considering multiple environments with different doors types. Furthermore, our dataset contains also negative images to model the typical uncertainty in which a robot operates. 

The collected results show also that our doors detector, built using DETR \cite{detr}, is able to detect doors in a new environment with variable performance (especially the challenging category of closed doors). We demonstrate that the \textbf{one-shot incremental learning} is a valid approach for increasing the performance of a general doors detector to specialize in a specific environment. We also argue that a little fine-tune operation, considering the 25\% of examples of a new environment, is enough to obtain a significant performance improvement also for the more challenging environments. This is shown by Fig. \ref{fig:all_houses_closed}, where the AP scores over the closed doors significantly grow up from \textsf{GD} to \textsf{FD\textsubscript{25}} on \textsf{house2} and \textsf{house10}.



             % Product Prototype
\chapter{Conclusions and Future Works}
\label{sec:chapter6}
\thispagestyle{empty}

In this thesis, we have addressed the task of finding doors in autonomous mobile robots. To achieve this, we proposed a doors detector based on DETR, a deep end-to-end module that performs object detection in RGB images by exploiting the capabilities of a CNN backbone (ResNet) and a Transformer. To increase the model's performance in a specific environment, we also present a technique, called \textit{one-shot incremental learning}. This paradigm is based on the observation that often a deployed robot operates in a single environment for a long time. Following the \textit{wayfinding} principle, an artificial scene presents a coherent visual aspect as well as doors of a single environment expose similar features. The one-shot incremental learning technique proposed in this thesis aims to fine-tune a general door detector with new examples to specialize the module for a specific environment. We also propose an approach to acquire a visual dataset and an evaluation metric to better measure the performance of a Deep Learning-based object detection module used by a mobile robot. 

The results show that our door detector reaches good results on the collected dataset. Our dataset allows us to evaluate the detector's performance in multiple environments and to simulate the deployment scenario considered in our work. Furthermore, we demonstrate that the one-shot incremental learning is a valid approach for increasing the detection accuracy of an end-to-end module used by a mobile robot. The results we collect show that the performance of a generic door detector strongly depends on the visual characteristics of the new environment in which it operates. The one-shot incremental learning mitigates this issue. Another interesting outcome of our experiments is that the smaller fine-tune operation we perform (considering the 25\% of examples collected in a new environment) allows obtaining the best performance improvement. 

In conclusion, we hope that this work has provided valuable insights about applying Deep Learning to Robotics considering the requirements of the typical scenarios in which autonomous agents operate. Since this thesis represents a small step into this research line, we report the most significant extensions of our work to suggest future investigations.

We collect the doors' examples in virtualized environments that do not allow an autonomous navigation of a simulated robot. Exploring other simulation technologies that allow an autonomous exploration of a mobile robot is a crucial step to collecting a more realistic dataset which better represents how a robot perceives an environment. Another important evolution of our work regards testing our approach on the field to evaluate the impact of door detection in robotic tasks, such as navigation, exploration, and planning.

The semantic information provided by the worlds' dataset is not sufficiently accurate to perform an automated labeling procedure of the visual dataset we collect. In fact, the bounding boxes of our dataset have been designed by a human operator. A primal direction for future works is to consider other worlds' datasets that expose more refined semantic information to acquire a larger and more precise visual dataset.

The proposed doors detector is built using DETR, the first end-to-end architecture that performs object detection exploiting the Transformers peculiarities. A valid extension of our work is to evaluate the performance of such a model in low-powered devices to measure its inference speed when run by a mobile robot. In addition, comparing the detection and inference performance of DETR with other well-known detectors represents a valid forward step of our work. 

The one-shot incremental learning technique we propose consists of fine-tuning a general  doors detector with new examples collected in an environment not included in the initial training phase. In the real deployment scenario, a human operator has to collect the images and label them with a visual tool to design the ground truth bounding boxes. An extension of this work consists of using the general doors detector rather than manually annotating the new examples and evaluate the performance improvement of the general doors detector fine-tuned with new image labeled by itself.

Finally, another important improvement regards the evaluation metric. We test the detector 
accuracy on positive images considering 2 labels: closed door and open door. The Computer Vision metric we use does not consider the fact in which a door is detected but with a wrong label as well as it can not consider a simpler binary classification task (door or no door). Refining the evaluation metric or using another one that considers these facts is a valid step for future researches.




\chapter{Direzioni future di ricerca e conclusioni}
\label{capitolo7}
\thispagestyle{empty}

\begin{quotation}
{\footnotesize
\noindent\emph{``''}
\begin{flushright}
Pari e dispari
\end{flushright}
}
\end{quotation}
\vspace{0.5cm}

\noindent Si mostrano le prospettive future di ricerca nell'area dove si \`e svolto il lavoro. Talvolta questa sezione pu\`o essere l'ultima sottosezione della precedente. Nelle conclusioni si deve richiamare l'area, lo scopo della tesi, cosa \`e stato fatto,come si valuta quello che si \`e fatto e si enfatizzano le prospettive future per mostrare come andare avanti nell'area di studio.

\printbibliography
%**************************************************************
% Materiale finale
%**************************************************************
\end{document}
