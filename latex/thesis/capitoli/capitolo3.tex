\chapter{Problem Formulation}
\label{capitolo3}
\thispagestyle{empty}

\section{Motivation and Goal}

As mentioned in the previous chapter, doors are useful semantic features for an active agent. Collecting information about doors, a robot improves significantly its knowledge about the environment in which it operates. Exploration and navigation, two of the main task for a mobile robot, are strongly influenced by doors, especially considering their location and status (e.g. closed or opened). During the mapping procedure, doors detection provide useful information to the robot, usable to improve the exploration of an unknown environment. By detecting closed doors, an agent can guess that some floor areas are temporarily unreachable and it can act to collect entirely the environment map. For example, it can open the door through a robotic arm or asking help to a human operator. Another solution may be identifying the presence of unreachable location by detecting closed door and upgrade the incomplete environment map in future stages (when the closed door have been opened). open the temporarily suspend the mapping procedure and restart to a human operator to open the door or it can open it by itself  For example,  so it  Different ares of an indoor environment can be reachable or unreachable, depending by doors' status (a closed door is represents an onsstacle ) of the doors that  reaThe door's status can be useful to identify areas that can be ubreacable in future (an open door can be closed)An autonomous agent can also 