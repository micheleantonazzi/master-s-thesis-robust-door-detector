% !TEX encoding = UTF-8
% !TEX TS-program = pdflatex
% !TEX root = ../tesi.tex

%**************************************************************
\hypertarget{Introduzione}{%
	\chapter{Introduzione}\label{header-n3}}

\hypertarget{Contenuti del documento}{%
	\section{Contenuti del documento}\label{header-n4}}

Nelle pagine successive verranno trattati i seguenti argomenti:
\cite{7942676}
\begin{itemize}
	\item
	\textbf{Capitolo 2:} verrà descritta la realtà lavorativa all'interno
	della quale si è svolto lo stage. Dopo un'iniziale e generica
	descrizione, SICS S.r.l. sarà analizzata sotto vari aspetti al fine di
	delinearne un profilo completo. In particolare verranno elencati i
	prodotti commercializzati e verranno tracciati il \emph{business
		model}\textsubscript{G} e il \emph{way of working}\textsubscript{G}.
	Questo approfondimento iniziale ha lo scopo di tracciare un filo
	conduttore tra la teoria studiata in aula e l'applicazione concreta di tali concetti nel mondo del lavoro.
	\item
	\textbf{Capitolo 3:} in questa sezione verrà descritto il
	\emph{progetto}\textsubscript{G} di stage che lo studente ha
	intrapreso, accompagnato dalle motivazioni che hanno condizionato tale
	scelta.
	\item
	\textbf{Capitolo 4:} qui verrà descritto lo svolgimento del
	\emph{progetto}. Inizialmente esso verrà definito con precisione,
	riportando il piano di lavoro e i vincoli ai quali è
	sottoposto. Successivamente verranno elencati le tecnologie e gli strumenti utilizzati. Un'analisi
	approfondita verrà riservata al \emph{framework}\textsubscript{G}
	\emph{GStreamer}, indispensabile per lo sviluppo del software. In
	conclusione ci sarà una descrizione del software prodotto.
	\item
	\textbf{Capitolo 5:} in questa sezione saranno riportate le
	considerazioni finali, raccolte dallo studente una volta concluso il
	tirocinio. In particolare verranno elencate le conoscenze acquisite e i
	gli obiettivi aziendali e personali raggiunti.
\end{itemize}

\hypertarget{header-n15}{%
	\section{Norme tipografiche}\label{header-n15}}

Sono qui definite le norme tipografiche applicate all'intero documento,
atte a facilitare la lettura e la comprensione del testo.\\

All'inizio del documento è presente un \textbf{indice} che ne agevoli la
consultazione, permettendo una lettura ipertestuale oltre alla normale
consultazione sequenziale. Deve essere presente anche l'indice delle
tabelle e l'indice delle figure, entrambi completi di numero
identificativo e di una breve descrizione dell'elemento.\\

Al fine di facilitare la comprensione del documento viene fornito un
\textbf{glossario}, contenente le definizioni di termini specifici o
ambigui. Ognuno di questi termini è marcato nel testo utilizzando il
carattere corsivo e una 'G' a pedice.\\

Il carattere \textbf{grassetto} viene utilizzato nei titoli e per
evidenziare gli elementi di un elenco puntato mentre il carattere
\textbf{corsivo} evidenzia le citazioni, le parole inserite in glossario
e i termini particolari.\\

Per identificare i \emph{requisiti}\textsubscript{G} propri del
\emph{progetto} verrà utilizzata la seguente sigla:

\centerline{\textbf{R{[}X{]}.{[}Y{]}.{[}ZZ{]}}}
In cui:

\textbf{X} rappresenta l'importanza strategica del \emph{requisito}, che
può essere:

\begin{itemize}
	\item
	\textbf{0:} \emph{requisito} obbligatorio;
	\item
	\textbf{1:} \emph{requisito} desiderabile;
	\item
	\textbf{2:} \emph{requisito} opzionale. 
\end{itemize}

\textbf{Y} rappresenta la tipologia del \emph{requisito} o e può
assumere i seguenti valori:

\begin{itemize}
	\item
	\textbf{F:} \emph{requisito} funzionale;
	\item
	\textbf{Q:} \emph{requisito} qualitativo; 
	\item
	\textbf{V:} \emph{requisito} di vincolo. 
\end{itemize}

\textbf{ZZ} è codice progressivo a due cifre che identifica univocamente
il \emph{requisito}.